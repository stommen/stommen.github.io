\chapter*{Attended Courses}
\addcontentsline{toc}{chapter}{Attended Courses}
In general, I'm happy with my course selection, as I believe that I found the perfect fraction between amount of workload and interesting courses. I had to drop some courses that didn't fit my schedule, but I think that I found good alternatives. Generally, I believe that NTU offers a wide range of interesting courses in many areas, so a great tip (especially for students from MPPHS, where the mandatory courses are done in LP1) is to check out courses from for example Nanyang Business School (NBS), National Institute of Education (NIE) or other colleges depending on your interests. 

Regarding the optimal schedule, most exchangers try to minimize the amount of classes close to the weekends (for traveling). However, I had at least one lecture every day, but the courses on Thursdays and Fridays often didn't require attendance. So, a tip here is to contact the lecturer and ask if the classes are mandatory before adding a course to your schedule. From my experience, most lecturers answer quickly and are very helpful. I do believe that the attendance policy differ much between the different colleges, where courses from e.g. NBS often have mandatory attendance and courses from SPMS often don't. Nevertheless, keep in mind that if you have classes on, say Thursdays or Fridays, there might be mandatory assessments on these days. I experienced this in PH4418, where I had a test every 3rd week on Thursdays. Also, I'd recommend to minimize the number of exams. That way, you do more continuous assessment and have more time to travel in the end of the semester. So, in short, I'd say that an optimal schedule for a travel-interested student would be to have classes on Mondays, Tuesdays and Wednesdays, and then have the rest of the week free, while having as few exams as possible.

As I was the first MSc student from MPPHS at NTU ever (or at least for a long time), I hadn't much to go on regarding relevant courses for my program. However, the \href{https://www.ntu.edu.sg/spms/about-us/physics/undergrad/course-info}{College of Science website} provided very good information about the courses available, so it wasn't an issue to find interesting courses within the physics subject. Lastly, the accreditation was made such that 1 AU = 2 hp. In practice, this was done by logging on to the \href{https://mex.portal.chalmers.se}{Chalmers MEX portal} and create a request for accreditation of your attended courses. Here, you had to attach the syllabus of each course along with your transcript of records from NTU. For the latter, I used the \href{https://wis.ntu.edu.sg/webexe/owa/request_transcript_und.LoginN?pType=SH}{Self-help Transcript System} to generate a non-signed version of this when I'd gotten my grades in the beginning of June (I had to meet my Director of Studies in person to show how I retrieved this since it wasn't signed). If you're more patient than me, NTU also sends you an email with an official, signed version in the end of June (I also believe that your Chalmers coordinator receives a hard-copy of this, if this email doesn't reach you). However, your request was later reviewed (and hopefully approved) by your Director of Studies, see \href{https://www.chalmers.se/utbildning/dina-studier/tjanster-och-verktyg/mex-min-vag-mot-examen/}{this page} for more information.
\newpage
\section*{Course Descriptions and Impressions}
\phantomsection
\addcontentsline{toc}{section}{Course Descriptions and Impressions}
% Below, I've briefly described the courses I attended during my exchange semester at NTU, along with my impressions of each course.

% \vspace{-0.3cm}
% \hrulefill

{\large{\textbf{Open Quantum Systems $\vert$ PH3406} \hfill \textbf{4 AU}}}

\vspace{-0.3cm}
\textbf{Accredited as: \hfill Exam: Yes}

{\large{\textsc{Lecturer/Examiner:} \textit{Dr. Koh Teck Seng}}} \vspace{0.3cm}

{\large{\textbf{\textit{Content}}}} 
\vspace{-0.25cm}

This course covers key concepts, including density matrices, the interaction picture, the Born-Markov and Lindblad master equations, decoherence, relaxation, dephasing, and linear dynamical maps. It explores canonical models (collisional, spin-boson, spin-spin) and connects quantum error correction to the Kraus representation theorem.

{\large{\textbf{\textit{Impression}}}}
\vspace{-0.25cm}

I really enjoyed this course, as it was a good mix between theory and practical applications. Dr. Koh is very nice and a great lecturer, and he's very passionate about the subject. The course was very well organized and assessed through one homework assignment, one project report, a midterm and an exam in the end. The homework and project were great for understanding the material and involved coding, which made the course closer to quantum computing, which I found very interesting. The midterm and exam were quite hard, but I only strived for a pass, so the amount of extra work I put down was more from my own interest than from necessity. I'd recommend this course to anyone interested in quantum mechanics, as it is a great introduction to the open quantum systems.

\hrulefill

{\large{\textbf{Physics in the Industry $\vert$ PH4418} \hfill \textbf{4 AU}}}
\vspace{-0.3cm}

\textbf{Accredited as: \hfill Exam: No}

{\large{\textsc{Lecturer/Examiner:} \textit{Prof. Lew Wen Siang \& Dr. Leek Meng Lee}}} \vspace{0.3cm}

{\large{\textbf{\textit{Content}}}} 
\vspace{-0.25cm}

This course explores major industries in Singapore where physics plays a key role. It provides an overview of diverse applications, helping students understand career opportunities and make informed employment decisions after graduation.

{\large{\textbf{\textit{Impression}}}}
\vspace{-0.25cm}

This wasn't one of my favorite courses. The course is divided into four topics: semiconductors, photonics, food physics and biomedical physics. Each topic is taught by a different lecturer, and the course is assessed through an in-class test after each topic and a final report and presentation (which can be done in pairs). The tests were rather easy, however the final report and presentation demanded some extra work. However, the topics and lecturers didn't catch my interest, and I didn't learn much from the course. I'd probably only recommend this course as a filler course, as the workload isn't too high.

\newpage

{\large{\textbf{Leadership in the 21st Century $\vert$ BU5642} \hfill \textbf{3 AU}}}
\vspace{-0.3cm}

\textbf{Accredited as: \hfill Exam: No}

{\large{\textsc{Lecturer/Examiner:} \textit{Assoc Prof. Jing Zhu}}} \vspace{0.4cm}

{\large{\textbf{\textit{Content}}}} 
\vspace{-0.25cm}

The Leadership in the 21st century course takes an evidence-based approach to leadership theory, frontier research, and real-world applications. Through interactive seminars with role-plays, case studies, and discussions, students develop leadership skills for managerial, consulting, or personal growth.

{\large{\textbf{\textit{Impression}}}}
\vspace{-0.25cm}

Together with HY0001, this course provided me MTS-credits back home at Chalmers, which was very convenient for me. However, in contrary to HY0001 (in my opinion), this course actually contains very valuable and important content. The lecturer, Jing, is very driven, kind and wants everybody to succeed and benefit from the course. I believe that everyone would benefit from this course, even though they don't aspire to become leaders in the future. Much of the content can be boiled down to ``knowing and understanding people'', however often in a setting where leadership plays a role. If you put your whole heart and effort into this course, I believe that you will grow much, both as a person and as a leader, and you'll be well-equipped for your future career when it comes to leadership. The assessment consisted of two quizzes and a team project with a written report and presentation, and I wouldn't say that the workload was heavy in any way.

\hrulefill

{\large{\textbf{Ethics and Moral Reasoning $\vert$ HY0001} \hfill \textbf{1 AU}}}
\vspace{-0.3cm}

\textbf{Accredited as: \hfill Exam: No}

{\large{\textsc{Lecturer/Examiner:} \textit{Jacob Mok}}} \vspace{0.4cm}

{\large{\textbf{\textit{Content}}}} 
\vspace{-0.25cm}

This course introduces key moral values like benevolence, impartiality, and integrity through major ethical theories. It fosters critical thinking on complex moral issues, explores academic integrity and research ethics, and examines the role of ethics in environmental sustainability.

{\large{\textbf{\textit{Impression}}}}
\vspace{-0.25cm}

There's really not much to say about this course. All content was delivered online, even the assessments. You need to go through a number of lectures, where each lecture ends with a quiz. The quizzes are very easy, and you can do them up to three times using all available resources. Except for the quizzes, you also need to write a short text on one of the topics, and a peer-review on another student's text. From my experience, most of the content in the course fall under the ``common sense'' category, and I didn't learn much from the course. However, it was a good filler course, and it also gave me MTS credits at Chalmers, which was very convenient.

\newpage

{\large{\textbf{C \& C++ Programming $\vert$ SC1008} \hfill \textbf{3 AU}}}
\vspace{-0.3cm}

\textbf{Accredited as: \hfill Exam: No}

{\large{\textsc{Lecturer/Examiner:} \textit{Assoc Prof. Hui Siu Cheung}}} \vspace{0.4cm}

{\large{\textbf{\textit{Content}}}} 
\vspace{-0.25cm}

This course introduces foundational concepts in C and C++, focusing on system programming, embedded systems, and performance optimization. It covers applications in gaming engines, virtual reality, web browsers, databases, and blockchain technology.

{\large{\textbf{\textit{Impression}}}}
\vspace{-0.25cm}

This was an introductory course to C and C++ programming. The lectures were conducted online, and there were also a mandatory lab/tutorial session each week. The course was divided in one C and one C++ part. Each part contained assessed assignments that were due every other week (all tools were allowed for these). Apart from these assignments, the course was assessed through a final test at the end of each part. For the C-part, this contained both coding and multiple choice questions (MCQs), and for the C++ only MCQs. I'd probably only recommend this course for two reasons -- either if you like programming and want to add to your repertoar, or as a filler course, as the level is basic and the organization fits the exchanger-life very well.

\hrulefill

{\large{\textbf{Rugby $\vert$ SS5205} \hfill \textbf{3 AU}}}
\vspace{-0.3cm}

\textbf{Accredited as: \hfill Exam: No}

{\large{\textsc{Lecturer/Examiner:} \textit{Harrie Desianto Hussien}}} \vspace{0.4cm}

{\large{\textbf{\textit{Content}}}} 
\vspace{-0.25cm}

This rugby course teaches fundamental skills, strategies, and game principles through tag and touch rugby. It combines theory and practice using the Sport Education and Games Concept Approach (GCA) to enhance understanding and gameplay competency.

{\large{\textbf{\textit{Impression}}}}
\vspace{-0.25cm}

This course was one of my favorites during the semester. And before you reject the course due to the physicality of usual rugby, you should know that this was a non-contact course (touch rugby). This changes the game a lot, as you don't have to worry about receiving brutal tackles or getting injured (though I kind of looked forward to the physical aspects, so if you're into that, you might want to look for a different course for that reason). However, it was probably for the best to avoid injuries while abroad, and I still believe that you'll learn a lot about rugby as a sport, as contact- and touch rugby are similar in many ways. We met once a week for three hours, and the course was assessed through an easy theoretical quiz and a practical assessment that took place during the last two classes. In the beginning, the focus is more on ball handling, rules and tactics. Then, during the last four to five weeks, each lesson basically only contains game-play with short breaks in-between. Furthermore, Coach Harrie is very nice, experienced and passionate about the sport. I'd definitely recommend this course to anyone interested in sports, especially on exchange, as it's fun, social and comes with minimal workload.
\newpage

\section*{Course Rankings}
\phantomsection
\addcontentsline{toc}{section}{Course Rankings}

\begin{center}
    \hrulefill {\Large\textsc{ Open Quantum Systems}} \hrulefill
    \vspace{0.5cm}
\renewcommand{\arraystretch}{1.5}
\begin{tabular}{>{\centering\arraybackslash}p{0.23\textwidth} >{\centering\arraybackslash}p{0.23\textwidth} >{\centering\arraybackslash}p{0.23\textwidth} >{\centering\arraybackslash}p{0.23\textwidth}}
    \large{\textbf{Workload}} & \large{\textbf{Content}} & \large{\textbf{Teaching}} & \large{\textbf{Total}} \\
    \ratingsquare{3}{5} & \rating{5}{5} & \rating{5}{5} & \rating{5}{5} \\ 
\end{tabular}
\end{center}
\vspace{-0.7cm}
\hrulefill

\vspace{0.7cm}

\begin{center}
    \hrulefill {\Large\textsc{ Rugby}} \hrulefill
    \vspace{0.5cm}
\renewcommand{\arraystretch}{1.5}
\begin{tabular}{>{\centering\arraybackslash}p{0.23\textwidth} >{\centering\arraybackslash}p{0.23\textwidth} >{\centering\arraybackslash}p{0.23\textwidth} >{\centering\arraybackslash}p{0.23\textwidth}}
    \large{\textbf{Workload}} & \large{\textbf{Content}} & \large{\textbf{Teaching}} & \large{\textbf{Total}} \\
    \ratingsquare{1}{5} & \rating{4}{5} & \rating{5}{5} & \rating{5}{5} \\ 
\end{tabular}
\end{center}
\vspace{-0.7cm}
\hrulefill

\vspace{0.7cm}

\begin{center}
    \hrulefill {\Large\textsc{ Leadership in the 21st Century}} \hrulefill
    \vspace{0.5cm}
\renewcommand{\arraystretch}{1.5}
\begin{tabular}{>{\centering\arraybackslash}p{0.23\textwidth} >{\centering\arraybackslash}p{0.23\textwidth} >{\centering\arraybackslash}p{0.23\textwidth} >{\centering\arraybackslash}p{0.23\textwidth}}
    \large{\textbf{Workload}} & \large{\textbf{Content}} & \large{\textbf{Teaching}} & \large{\textbf{Total}} \\
    \ratingsquare{2}{5} & \rating{3}{5} & \rating{5}{5} & \rating{4}{5} \\ 
\end{tabular}
\end{center}
\vspace{-0.7cm}
\hrulefill

\vspace{0.7cm}

\begin{center}
    \hrulefill {\Large\textsc{ C \& C++ Programming}} \hrulefill
    \vspace{0.5cm}
\renewcommand{\arraystretch}{1.5}
\begin{tabular}{>{\centering\arraybackslash}p{0.23\textwidth} >{\centering\arraybackslash}p{0.23\textwidth} >{\centering\arraybackslash}p{0.23\textwidth} >{\centering\arraybackslash}p{0.23\textwidth}}
    \large{\textbf{Workload}} & \large{\textbf{Content}} & \large{\textbf{Teaching}} & \large{\textbf{Total}} \\
    \ratingsquare{2}{5} & \rating{2}{5} & \rating{2}{5} & \rating{2}{5} \\ 
\end{tabular}
\end{center}
\vspace{-0.7cm}
\hrulefill

\vspace{0.7cm}

\begin{center}
    \hrulefill {\Large\textsc{ Physics in the Industry}} \hrulefill
    \vspace{0.5cm}
\renewcommand{\arraystretch}{1.5}
\begin{tabular}{>{\centering\arraybackslash}p{0.23\textwidth} >{\centering\arraybackslash}p{0.23\textwidth} >{\centering\arraybackslash}p{0.23\textwidth} >{\centering\arraybackslash}p{0.23\textwidth}}
    \large{\textbf{Workload}} & \large{\textbf{Content}} & \large{\textbf{Teaching}} & \large{\textbf{Total}} \\
    \ratingsquare{2}{5} & \rating{2}{5} & \rating{2}{5} & \rating{2}{5} \\ 
\end{tabular}
\end{center}
\vspace{-0.7cm}
\hrulefill

\vspace{0.7cm}

\begin{center}
    \hrulefill {\Large\textsc{ Ethics and Moral Reasoning}} \hrulefill
    \vspace{0.5cm}
\renewcommand{\arraystretch}{1.5}
\begin{tabular}{>{\centering\arraybackslash}p{0.23\textwidth} >{\centering\arraybackslash}p{0.23\textwidth} >{\centering\arraybackslash}p{0.23\textwidth} >{\centering\arraybackslash}p{0.23\textwidth}}
    \large{\textbf{Workload}} & \large{\textbf{Content}} & \large{\textbf{Teaching}} & \large{\textbf{Total}} \\
    \ratingsquare{0}{5} & \rating{2}{5} & \rating{1}{5} & \rating{2}{5} \\ 
\end{tabular}
\end{center}
\vspace{-0.7cm}
\hrulefill
