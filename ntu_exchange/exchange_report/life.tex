\chapter*{Life at NTU}
\phantomsection
\addcontentsline{toc}{chapter}{Life at NTU}
Living at NTU and in Singapore is a great experience (and adventure), and I'd say that it's a perfect place for an exchange. The campus is huge and beautiful with a lot of nature built into it and the city is very clean and safe. Moreover, the locals are friendly and the food is amazing. Lastly, NTU and Singapore as a whole accepts \textit{a lot} of exchangers, which enables you to not only experience and learn about the Singaporean culture, but cultures from all over the world. I met people from essentially every corner of the world. However, there are some things that you might want to know before going here, which I'll try to cover in this chapter. Hopefully, it will decrease the culture shock (if you get one) and make you more prepared for your studies at NTU.
\vspace{-0.2cm}
\section*{Everyday Life}\label{everyday}
\phantomsection
\addcontentsline{toc}{section}{Everyday Life}
As said, you'll sooner or later find your personal routine at NTU, despite being on the other side of the globe. For me, I believe that I found mine after some weeks or so, but I'm sure that this varies depending on how used you are to changes. A general tip is to find friends (either locals, other exchangers or Swedes) so that you can navigate the new surroundings together, which will be more fun and probably easier. Many things will of course not be that different from your usual life back home. Nevertheless, there are things that is substantially different from Sweden.

For starters, Singapore is \textit{hot}. The temperature during day is usually around 30 -- 35$^\circ$C, and the humidity is very high. The sun is also very strong, so sunscreen and water are your new best friends. On the contrary, the evenings are nice with a temperature around 25$^\circ$C, and you'll notice that the locals generally are more nightly active, most likely due to the climate. Eventually, you'll probably get somewhat used to the heat, although not like the locals who consider 28$^\circ$C ``cold''. Moreover, the weather is also very unpredictable, and the rain can be very heavy, but usually doesn't last for long. For this reason, I would recommend to always bring an umbrella. To escape the heat, you'll probably spend a lot of time in the AC. At NTU, many canteens along with most study rooms and libraries are air-conditioned, which initially made these my favorite places on campus.

Secondly, the cooking culture is completely different from Sweden. As I will explain in the section about Singapore as a whole, a large part of the food culture is based on the so called hawker centers (or canteens at NTU). These are basically large food courts, where you can find dishes from all over the world. This makes it cheap and easy to eat out, and I would say that most students at NTU take advantage of this. 

In fact, when living on campus, the facilities for cooking are very limited, and I would say that it's not worth the effort to cook. However, if you're a fan of cooking, you can always use the pentry in your hall, which has some hotplates and a microwave. For the different canteens at NTU, I would of course recommend to test them all and find your own favorite (the same goes for the dishes). Generally, I would say that the food is cheaper in hall canteens, while the more public ones are more expensive but have a wider range of dishes. Below, I've listed my favorite canteens at NTU:
\vspace{-0.3cm}
\begin{itemize}
    \item[--] \textbf{North and South Spine Canteens}: The largest canteens on campus that usually are close to where you have lectures. They have a wide range of dishes, but are often crowded and a bit more expensive. Most stalls here accept cashless payments.
    \item[--] \textbf{North Hill Canteen}: A wide range of nice dishes and usually not that crowded. Here, you must rely a bit more on cash.
    \item[--] \textbf{Canteen 2}: Essentially in the middle of campus with (in my opinion) some of the best meals. It is also cheaper than the North/South Spine canteens, with the downside that no stall accepts cashless payments.
    \item[--] \textit{Honorable Mention}: The Indian food at my canteen (Tamarind hall). This explains itself, probably the best indian food on campus (and my personal favorite).
\end{itemize}
\vspace{-0.3cm}
\begin{figure}[H]
    \centering
    \includegraphics[width=0.99\textwidth]{figs/food.pdf}
    \caption{\raggedright Bottom right: Canteen 2 (one of my favorite canteens). Center displays my go-to meal at campus -- a chinese chicken-noodle dish. Soups (in all forms) are also common to see. Bottom left is another personal favorite which seems weird at first -- chicken omelette rice. Lastly, top right is my favorite dish, Indian at Tamarind Hall canteen.}
\end{figure}
\vspace{-0.6cm}
Lastly, as campus is like a small town, there are many small grocery stores and shops where you can buy essentials, for example close to canteen 2 and in North Spine. For accessing larger malls and grocery stores, you can either take the free campus bus to Pioneer MRT station, where you can take the MRT to Jurong East. However, a closer option is to take the 199 bus from campus, which takes you to Jurong Point -- a large mall with a lot of stores and food stalls.
\section*{Housing}
\phantomsection
\label{house}
\addcontentsline{toc}{section}{Housing}
\vspace{-0.2cm}
After I was accepted to NTU, I applied for campus housing in the second half of November. Regarding living on- or off-campus, I would recommend to peek at some of the previous travel reports (e.g. that from Filip in 2022/23). However, in short I can say that I'm 110 \% happy with my decision to live on-campus. In my opinion, these are the main pros and cons with each option:
\begin{table}[H]
    \begin{center}
        \begin{tabular}{||c c | c c||}
            \toprule
            \multicolumn{2}{||c|}{\textbf{Living On-campus}} & \multicolumn{2}{c||}{\textbf{Living Off-campus}} \\ \toprule
            \textsc{Pros} & \textsc{Cons} & \textsc{Pros} & \textsc{Cons} \\ \hline\hline
            Cheaper & \makecell{Room \& facility \\ standard lower} & \makecell{Better facilities \\ (not guaranteed)} & More expensive \\ \midrule
            Closer to NTU & Far from downtown & Closer to downtown & Far from NTU \\ \midrule
            \makecell{Most exchange \\ students live here} & -- & -- & \makecell{More complicated} \\ \midrule
            \makecell{Engage more in \\ the NTU life} & -- & -- & -- \\ \bottomrule
        \end{tabular}
    \end{center}
\end{table}
\vspace{-1cm}

So, I think that you will meet more exchangers and soak up more of the NTU student life (e.g. eating in the canteens and engaging in hall activities) living on-campus. Of course, the room and facility standard would most likely be better living off-campus, but more pricy. Moreover, I would say that the procedure of getting an off-campus residence is in general more complicated than getting it on-campus. Before I get into my own housing, I'd like to say that I didn't hear about one person that applied for campus housing and didn't get it, since I know that this can be a concern for many. 

\begin{wrapfigure}{r}{0.31\textwidth}
    \centering
    \vspace{-0.67cm}
    \includegraphics[width=0.27\textwidth]{figs/room.pdf}
    \vspace{-0.4cm}
    \caption{My room in Tamarind Hall.}
    \vspace{-0.4cm}
\end{wrapfigure}
Now, I applied for a single room with AC and was assigned this at Tamarind Hall (or Tama) in the north-west of campus. I heard some weird stories from my friends who applied for single rooms and was assigned a double and vice versa, but this must be very rare. I was satisfied with my dorm and hall, which is one of the newest at NTU. For the whole semester, I paid around 20\,000 SEK, which isn't cheap considering the standard, but much cheaper than living off-campus (also, the AC costs around 0.4 SGD per usage hour). However, from my experience, the standard is more or less comparable all over campus. You'll probably see lizards, bugs and cockroaches independently of where you live, and eventually you'll probably get used to it. The shared bathrooms are maybe not of first class either, but I never had any trouble using them. In my hall (and I think that this applies to most of them), we also had access to a small pentry with cold and hot water and some hotplates. There were also washing and drying machines on certain floors, where each program costed 1 SGD-coin (so make sure to keep these if you get them). For the latter, you can either ask to get some extra 1ers when buying food, if you for example pay a little extra. Or, if you have a lot of small coins laying around, you can put these in the vending machines, cancel the purchase and hope that it spits out some 1ers (which it often does).
\begin{figure}[H]
    \centering
    \includegraphics[width=0.99\textwidth]{figs/fac1.pdf}
    \caption{\raggedright The shared bathroom and pentry on my floor.}
\end{figure}
\vspace{-0.6cm}
Something I've not seen in previous travel reports (maybe for obvious reasons) is the problem with mould in the halls. This is a common problem in Singapore due to the high humidity, and I along with many other exchangers experienced issues with this during the first month. For example, after coming home from a week of traveling, some of my clothes (!) had white mould. I reported this to my hall and they sent two women to ``educate'' me on the Singaporean climate. However, their tips actually seemed to work, as I didn't get much more mould after that. Their advice was to keep the ceiling fan off while keeping the window and ventilation open when being away for a longer period of time. As for the clothes, I used vinegar and hot water to clean them, and then I washed them using regular detergent with some vinegar in the washing machine, which did the trick. You could also buy a dehumidifier (I didn't though), which should lower the humidity in your room. I know that some students bought these at North Spine, otherwise you can probably find them at the malls in Jurong East or Jurong Point.
\section*{Economy}
\phantomsection
\label{eco}
\addcontentsline{toc}{section}{Economy}
\begin{wrapfigure}{l}{0.3\textwidth}
    \centering
    \vspace{-0.7cm}
    \includegraphics[width=0.29\textwidth]{figs/economy.pdf}
    \vspace{-0.1cm}
    \caption{\raggedright The large banking district in downtown Singapore.}
    \vspace{-0.4cm}
\end{wrapfigure}
Relative to other Southeast Asian countries, Singapore is expensive. Clothes, electronics and groceries are approximately at the save price level as in Sweden, if not higher in some cases. Alcohol is very expensive at bars and clubs, but there are areas downtown (e.g. Boat Quay, similar to 2a lång) where you can get a beer for about 50 SEK. When it comes to food, you already know about the hawker centers, where you can get a proper meal for around 30 -- 50 SEK. Only NTU has 10 -- 15 of these.

Nevertheless, I'd argue that the greatest concern and difference from Sweden is the cash situation. This is mostly an issue at hawker centers, where many food stalls don't accept cashless payments. When completing your STP, you'll receive your \textit{matriculation card} (essentially like the Chalmers kårkort). This has a so called NETS-functionality (a payment type in Singapore) which can be used at many places where card payment isn't accepted. This can be topped up through an app, making it a good alternative to cash. 

However, I went for an alternative solution, which I believe is the best one -- namely opening a Singaporean bank account, which I'll explain below. This way, you don't have to worry about topping up your matriculation card and you can pay directly from your phone. Before opening my bank account, I solely relied on cash and my \href{https://wise.com}{Wise}-card. This works smoothly, however if you're as allergic to cash as I am, I'd recommend you to get a bank account as soon as possible. Regarding Wise or \href{https://www.revolut.com/sv-SE/}{Revolut} (which seems like the two alternatives for the best exchange rates and lowest fees), I had a great experience using Wise during my time. However, there are pros and cons with both (although for me, there were more pros with Wise) and I know that most Swedish exchangers used Revolut.
\subsection*{Opening a Singaporean Bank Account}
\phantomsection
\addcontentsline{toc}{subsection}{Opening a Singaporean Bank Account}
Once you have got your STP approved, you can apply for Singpass through \href{https://portal.singpass.gov.sg/home/ui/register/instructions}{this} link. This is essentially the BankID of Singapore and you should after approval be able to reach this thorugh the \href{https://www.google.com/url?sa=t&source=web&rct=j&opi=89978449&url=https://apps.apple.com/sg/app/singpass/id1340660807&ved=2ahUKEwj8ouKcoaWLAxVyUGcHHd3CKWgQFnoECBUQAQ&usg=AOvVaw2MU54ywzShBUAGdsGy2E40}{Singpass}-app. Then, you can easily apply for a bank account at \href{https://www.dbs.com.sg/index/default.page}{DBS}, one of the major banks in Singapore (there are other options, but I don't know if they require more documents). This was done in the \href{https://www.google.com/url?sa=t&source=web&rct=j&opi=89978449&url=https://apps.apple.com/sg/app/dbs-digibank/id1068403826&ved=2ahUKEwjWjsikoqWLAxW2UGwGHWeHHRsQFnoECBcQAQ&usg=AOvVaw3MmyWfIWdIEfDqXsxZtKcG}{DBS digibank}-app, which will be where you manage your balance, make payments and so on. Apart from Singpass, you'll need a proof of your residential address in Singapore, which you can get from your hall's office. I also had to provide proof of my Swedish tax liability (I used an English Personbevis from Skatteverket). Then, your application will be processed for a few days before you get a confirmation email from DBS. They will also send a debit card to your address. 

This makes your life much easier, as you also get access to the Singapore equivalent of Swish -- PayNow. Like the NETS-functionality of the matriculation card, PayNow works via QR codes everywhere in Singapore. Another bonus of getting a bank account in Singapore is that if you do need cash, the withdrawals will be free, unlike if you use your Swedish card. DBS also provides some nice features when traveling, as you can open accounts in different currencies and transfer money between them without any fees (similar to Revolut and Wise). To transfer money from your Swedish account to your Singaporean account, I would recommend using Wise (Revolut is also an option), as they have the best exchange rates and lowest fees.

As a side note, you might experience issues while trying to type decimal numbers when using PayNow. In this case, change your phone's decimal format to ``.'' instead of ``,''. On an iPhone, this can be done under Allmänt $\rightarrow$ Språk och region $\rightarrow$ Numeriskt format.
\section*{Education -- Chalmers vs. NTU}
\phantomsection
\addcontentsline{toc}{section}{Education -- Chalmers vs. NTU}
When studying at NTU, you will definitely notice many similarities to Chalmers. All in all, you'll still be a university student and once you hit your daily routines, you'll probably almost forget that you're on the other side of the world. However, as soon as you escape the classroom or your dorm (and the AC), there will definitely be some differences.
\\
\\

To start with, the NTU campus is huge (and beautiful), and the surroundings make you feel like you're in the middle of the jungle, with large trees and plants. In the beginning of your exchange, you'll definitely be overwhelmed by the size and use Google Maps \textit{a lot}. As time goes on though, you'll find that it isn't that large and that you often can get around by foot. Although walking is possible, the heat (or rain) often makes the free bus system, which run frequently throughout the day (less frequent in the evenings) a better alternative. Despite the availability of buses, they tend to be very crowded during school hours, making walking a viable alternative if the distance isn't too long.

\begin{figure}[H]
    \centering
    \includegraphics[width=0.99\textwidth]{figs/environment.pdf}
    \caption{\raggedright The campus voted as one of the most beautiful in the world: Gaia (NBS building), The Hive, Yunnan Garden, walkway to my resident hall and the view from this.}
\end{figure}
\vspace{-0.6cm}
A small disclaimer is that Singapore's new MRT line, Jurong Region Line, is under construction. This makes a lot at campus look like a construction site, although it doesn't affect the everyday life at campus much. The good news is that future students won't be equally dependent on the MRT station at Pioneer (which is only accessible by bus) in order to go downtown, as the construction is planned to be finished in 2029. 

Academically, NTU emphasizes continuous assessment, with midterms and quizzes playing a significant role in most courses. Unlike at Chalmers, where final exams often constitute a large percentage of the final grade, NTU's exams rarely account for more than 60 \% of the total course grade. This structure encourages students to engage with the material consistently throughout the semester rather than relying on final exams, which I actually believe is a better way of learning. Another key difference, which I'm not as happy with, is the academic calendar. At NTU, all courses run concurrently throughout the semester, without the distinct reading periods used at Chalmers. This structure requires students to manage multiple subjects simultaneously over a longer period, which I generally dislike. Regarding academics, the quality of teaching, course content, and overall difficulty level are generally good. However, given NTU's high global ranking, I initially expected an even higher level. In comparison, I find that many (if not all) courses at Chalmers are more challenging in terms of depth and difficulty.

\begin{figure}[H]
    \centering
    \includegraphics[width=0.99\textwidth]{figs/study_places.pdf}
    \caption{\raggedright Great study facilities: Lee Wee Nam Library and outside at North Spine Plaza.}
\end{figure}
\vspace{-0.6cm}
Also the grading system at NTU differs from that at Chalmers. The highest grade is A+, while the lowest grade is F. Grades are assigned based on a bell curve (though I'm not entirely sure that this was applied to exchangers), meaning that students are not only assessed on their individual performance but also in comparison to their peers. However, as Chalmers students in general can't transfer grades from NTU to Chalmers, there is essentially no need to aim for the highest grade (instead I believe you should focus on traveling, exploring new cultures and personal development).

Lastly, the student culture at NTU is notably different from that at Chalmers. There's a strong academic focus, with almost no partying. The student-led clubs are almost always of a more serious nature, often focused on academic or professional activities rather than the ``sexmästerier'' common at Chalmers. A unique aspect of student life is the hall culture. Each residential hall has its own traditions, events, and even sports teams. One of the most notable traditions is the Hall Olympiad, held in the autumn semester, where different halls compete against each other in various sports.
\section*{Free Time and Activities}
\phantomsection
\addcontentsline{toc}{section}{Free Time and Activities}
\begin{wrapfigure}{l}{0.45\textwidth}
    \centering
    \vspace{-0.53cm}
    \includegraphics[width=0.42\textwidth]{figs/gym.pdf}
    \vspace{-0.1cm}
    \caption{\raggedright The gym at my hall.}
    \vspace{-0.32cm}
\end{wrapfigure}
After school hours, there's a lot to do around campus. As I mentioned earlier, campus is like a small town, and there are many activities and events to engage in. For example, there are many sports facilities, such as gyms, swimming pools, and sports courts, which are free for students. During my exchange semester, the only swimming pool open was the one at NIE, which was nice and never felt too crowded. Concerning gyms, many halls have their own (including my hall), but these are mostly small with limited equipment. The largest gym easily accessible, which I used mostly during the semester, is the one at North Hill. However, this gets really crowded during the evenings (though not to the degree of it being useless). You could also look into getting a membership at a gym outside campus, but this would be a bit of a hassle. 

Regarding sport facilities, you can engage in most sports at NTU. I frequently played table tennis and badminton during my time. However, you often need to book your time (can be done \href{https://www.google.com/url?sa=t&source=web&rct=j&opi=89978449&url=https://ntu.facilitiesbooking.com/&ved=2ahUKEwi-k6Pays2LAxUg2TgGHZjtK4sQFnoECAsQAQ&usg=AOvVaw0UI5Z1OVQuEv2rAMKN2LGh}{here} after logging in), and for some sports (especially badminton), available times were rare. Nevertheless, there are also many sports clubs at NTU, which you can join for a small fee. I joined the NTU Floorball Club due to my history with the sport, and they arranged game-play two times a week. This was a great way to meet locals while also maintaining my floorball skills. There's also many clubs outside of sports, such as music or dancing. Most halls also arrange different activities for its residents. Also, as an exchange student, there are many events arranged by NTU in the start of the semester, which I definitely recommend (mostly to meet other exchangers). There are also many parties arranged for exchangers (and non-exchangers) in Singapore. For this, you should use the \href{https://aentry.app/events}{Aentry} app, where you can find and buy tickets to many bars and clubs. Lastly, studying in Singapore of course is the perfect ground-zero spot for traveling in Asia, which I'll cover in the \hyperref[discover]{last chapter}.

\begin{figure}[H]
    \centering
    \includegraphics[width=0.99\textwidth]{figs/sports.pdf}
    \caption{\raggedright Great sports facilities: The running track at The Wave, the multi-purpose hall at North Hill with badminton courts and table tennis, the swimming pool and hockey pitch at NIE, and the volleyball court outside canteen 2.}
\end{figure}
