\chapter*{Discovering Asia and Singapore}
\phantomsection
\label{discover}
\addcontentsline{toc}{chapter}{Discovering Asia and Singapore}

One of the best things about studying abroad, and especially in Singapore, is the opportunity to travel and explore new places. As a country, Singapore doesn't have much to offer, which most locals also agree on. However, there are some nice places and things to do and explore, which I'll cover in the chapter. When asking local students, they often agree on that the best thing about Singapore is the location, making it easy to leave the country and explore the rest of Asia. This might sound strange and a bit sad, but after almost half a year in Singapore, I can agree on that. The country is small, but the location is perfect for exploring the rest of Asia, which I'll also cover in this chapter.
\section*{Singapore: City and Country in One}
\phantomsection
\addcontentsline{toc}{section}{Singapore: City and Country in One}
Even though the campus at NTU is large, beautiful and has a lot to offer, you'll eventually want to explore your new country (and city). Singapore is a small country, and you can easily explore most things in a few days. The public transport system is very good, and you can get around the city easily. The MRT (subway) is very cheap and easy to use, and you can also use buses to get around. You could get some sort of card for the public transport, but I would recommend to just use your bank card (or Apple/Google Pay) to pay for the rides. This is very convenient, and you don't have to worry about getting a card or reloading it. And of course, you can also use Grab (the Uber of South-East Asia) to get around, which is also cheap and available 24/7 in contrary to the public transport.

As for things to explore and see, there's some places that shouldn't be missed. Below, I've listed some of the most popular places to visit sorted by category.

\textbf{Nature:}
\vspace{-0.5cm}
\begin{itemize}
    \item \textbf{Sentosa Island:} A small island just south of Singapore, known for its beaches, resorts and attractions. You can easily spend a whole day here, and it's a great place to relax and enjoy the sun.
    \item \textbf{Gardens by the Bay:} A park with futuristic gardens and structures. The Supertree Grove is a must-see, especially at night when the trees are lit up.
    \item \textbf{MacRitchie Reservoir:} A large nature reserve with hiking trails. Here, you can also find the famous treetop walk, which is a suspension bridge that gives you a great view of the forest.
\end{itemize}

\textbf{Culture:}
\vspace{-0.3cm}
\begin{itemize}
    \item \textbf{Chinatown:} A famous Chinatown with a lot of history and culture.
    \item \textbf{Little India:} A colorful neighborhood with a lot of Indian culture. Here, you can find temples, markets and a lot of great food.
    \item \textbf{Orchard Road:} A famous shopping street with a lot of malls and shops.
    \item \textbf{Raffles Hotel:} A historic hotel where you can find the famous Singapore Sling cocktail at the Long Bar.
\end{itemize}

\textbf{Hawker Centers:}
\vspace{-0.3cm}
\begin{itemize}
    \item \textbf{Maxwell Food Center:} A popular hawker center in Chinatown with a lot of great food. Here, you can find the famous Tian Tian Hainanese Chicken Rice.
    \item \textbf{Lau Pa Sat:} A historic hawker center in the central business district. Here, you can find a lot of great food and a lively atmosphere.
\end{itemize}

\textbf{Nightlife:}
\vspace{-0.3cm}
\begin{itemize}
    \item \textbf{Clarke and Boat Quay:} A popular nightlife area with a lot of bars. This is essentially the ``Andra-Lång'' of Singapore, with cheap beer and great vibes. The bar Georgetown is a popular place among exchange students.
    \item \textbf{Cè La Vie:} A popular nightclub with a great view of the city from the top of Marina Bay Sands (don't drink too many beers though, or you'll be ruined).
    \item \textbf{Zouk and Marquee:} Nightclubs.
\end{itemize}

\begin{figure}[H]
    \centering
    \includegraphics[width=0.98\linewidth, trim={0 0.3cm 0 0}, clip]{figs/sg.pdf}
    \caption{\raggedright Despite its small size, Singapore has some must-do activities and places to visit. Clockwise from the top left: Juwel Changi Airport, Gardens by the Bay, Sentosa Island, Buddha Tooth Relic Temple, Marina Bay Sands and the view from Cè La Vie nightclub.}
\end{figure}

As for the cultural aspects, Singapore is a very multicultural country with a lot of different cultures and religions. The official language is English, but you'll also hear a lot of Mandarin and other languages. For example, the locals speak a very special version of English called Singlish, which is a mix of English and some other local languages. This can be a bit hard to understand at first, but you'll get used to it. The country is also very safe, and I never felt unsafe or uncomfortable. This can for example be seen in school, where everyone leaves their laptops and bags unattended in the library or canteens, which is something that would rarely happen in Sweden. 
\vspace{-0.15cm}
\section*{Asia: A Student-Friendly Continent}
\phantomsection
\label{asia}
\vspace{-0.1cm}
\addcontentsline{toc}{section}{Asia: A Student-Friendly Continent}
In many ways, Asia is the perfect continent for students. The cost of living is usually low, the food is amazing and there's a lot to see and explore. When your base is Singapore, you can usually find a lot of options for cheap flights to many neighboring countries. I traveled to Malaysia, Thailand, Vietnam, Indonesia and Hong Kong during the school semester, and then Laos, Japan, South Korea and Taiwan before I went back to Sweden. Below, I have shortly noted my experiences and some tips about the countries I visited. Note that cash still is king in most Asian countries, so I would recommend to always have some cash with you, especially in the South-East Asian countries.

\textbf{Malaysia:} I only visited Kuala Lumpur for a short time, but I really liked the city, which is both modern and packs a lot of culture. Unfortunately, I missed the Batu Caves, which I've heard is a must-see, so try to make time for that when in KL. I also spent one day in Johor Bahru (JB), a city just across the border from Singapore, and it's a great place to visit for a day trip (this is actually many Singaporeans favorite weekend-destination for eating and shopping). Malaysia is also very cheap in general, and the food is amazing.

\begin{wrapfigure}{r}{0.25\textwidth}
    \centering
    \vspace{-0.55cm}
    \includegraphics[width=0.24\textwidth]{figs/thaibeach.pdf}
    \vspace{-0.1cm}
    \caption{\raggedright Beach on Koh Lanta.}
    \vspace{-0.32cm}
\end{wrapfigure}
\textbf{Thailand:} I went a week to Koh Lanta and Koh Phi Phi in Krabi for ``studying'' and relaxing. Thailand is one of my favorite countries -- it's cheap, the food is amazing, the people are very friendly and the weather and beaches are great. Apart from Koh Lanta and Koh Phi Phi, there's a lot of other alternatives to choose from, such as Phuket, Koh Samui and Koh Phangan. Nevertheless, I would recommend to visit Koh Lanta, which is a bit more quiet and relaxed than the other places. Koh Phi Phi is a bit more touristy, but has some amazing beaches and views. I've also heard that Bangkok is great, and also Chiang Mai in the north, which I didn't have time to visit.

\begin{wrapfigure}{r}{0.25\textwidth}
    \centering
    \vspace{-0.55cm}
    \includegraphics[width=0.24\textwidth]{figs/ninhbinh.pdf}
    \vspace{-0.12cm}
    \caption{\raggedright The landscape in Ninh Bình.}
    \vspace{-0.32cm}
\end{wrapfigure}
\textbf{Vietnam:} I was in Vietnam for a week during recess, and visited Hanoi, Ninh Bình and H\d{ô}i An. Vietnam is also one of my favorites, with friendly people, amazing food and a lot of culture. Hanoi is a very busy and lively city with culture, history and a lot of great food. Ninh Bình is more nature-focused, for example, we went on a kayak trip through the many caves and mountains in the area. H\d{ô}i An is a smaller historical town known for its many tailor-stores, perfect for getting custom-made clothes. If I had time, I also would've liked to visit Ho Chi Minh City in the south, known for its food and history from the Vietnam War.

\begin{wrapfigure}{r}{0.4\textwidth}
    \centering
    \vspace{-0.15cm}
    \includegraphics[width=0.39\textwidth]{figs/hk.pdf}
    \vspace{-0.18cm}
    \caption{\raggedright The Hong Kong Island skyline.}
    \vspace{-0.32cm}
\end{wrapfigure}
\textbf{Hong Kong:} I went to Hong Kong for a weekend, and I really liked it. In many ways, the city feels similar to Singapore, but I believe that HK has a lot more to offer and the city feels more alive. I visited Victoria Peak, which has a great view of the city. We also hiked along the Dragon's Back trail, which is a must-do when in HK (unfortunately, we didn't have the weather gods on our side). Other than that, I would recommend to explore the city and find your own favorite spots. We had a very interesting experience at Mr. Wong's, a restaurant in the city. For 100 HKD, you get to eat (and drink) as much as you want, and the place is popular among exchange students.

\begin{wrapfigure}{r}{0.29\textwidth}
    \centering
    \vspace{-0.53cm}
    \includegraphics[width=0.28\textwidth]{figs/nusapenida.pdf}
    \vspace{-0.1cm}
    \caption{\raggedright Kelingking Beach.}
    \vspace{-0.32cm}
\end{wrapfigure}
\textbf{Indonesia:} I went to Batam during a weekend, which is a small island only 45 minutes away from Singapore by ferry. As the rest of Indonesia, Batam is very cheap, and we had a great time just relaxing and studying in a large villa. I also visited Bali for a week. Here, I started in Ubud, known for its nature, many waterfalls, rice terraces and temples (try the river-rafting!), Nusa Penida, a small island with amazing views and beaches (Kelingking is the Instagram-king of beaches), and Seminyak, a beach town in the south. Here, I tried surfing which was a lot of fun (but hard). I also visited the famous Uluwatu Temple (watch out for the monkeys). If you want a mix of culture, relaxing and adventure, Bali is probably the perfect place.

\begin{wrapfigure}{r}{0.28\textwidth}
    \centering
    \vspace{-0.53cm}
    \includegraphics[width=0.27\textwidth]{figs/laos.pdf}
    \vspace{-0.1cm}
    \caption{\raggedright Hot air balloon tour in Vang Vieng.}
    \vspace{-0.35cm}
\end{wrapfigure}
\textbf{Laos:} My first stop after school was Laos. I believe that Laos is one of the most underrated countries in Asia, and I really liked it. It is very cheap, the people are very friendly and the nature is amazing. We flew to the capital Vientiane and spent a half day there (this city doesn't offer too much to see, so I'll probably wouldn't spend more than one or two days here). From here, we took a mini bus to Vang Vieng, a small town known for its nature and adventurous activities, where we spent three nights. Here, we went on a hot air balloon tour (this is a must-do!). We also rented a motorcycle for a cheap price (this is without a doubt the best way to explore the area) and visited some caves (where you can try water-tubing), blue lagoons (the best places to chill and cool off in the heat, there are several of these in the area) and hiking places. Lastly, we took the train to Luang Prabang, a UNESCO World Heritage Site known for its temples and culture, where we spent two nights. As a side-note, the railway is actually very good, modern and efficient in Laos, mostly due to the Chinese investment in the country. Here, we also rented a motorcycle and visited the Kuang Si Waterfalls, which is a must-see when in Laos. The waterfalls are beautiful, and you can also swim in the turquoise water. I also recommend the night market and the bowling alley, where you can play bowling and drink beer for a very cheap price.

\begin{wrapfigure}{r}{0.26\textwidth}
    \centering
    \vspace{-0.15cm}
    \includegraphics[width=0.25\textwidth]{figs/tokyo.pdf}
    \vspace{-0.1cm}
    \caption{\raggedright The view from Tokyo Skytree.}
    \vspace{-0.35cm}
\end{wrapfigure}
\textbf{Japan:} After Laos, I spent 10 days in Japan, where we visited, Tokyo, Mt. Fuji, Kyoto, Nara and Osaka. Japan is a very unique country, with a mix of modern and traditional culture. The public transport system is very good, and you can easily get around using the Shinkansen (bullet train) or the subway. For the subway, I would recommend the Suica card (which you for example can access through Apple Wallet, just tap ``add card -- travel card'' in the app). This can be topped up via Apple Pay, which is very convenient. We started with four nights in Tokyo, which, except for being the largest city in the world, is very busy and lively with a lot of culture and great food. Some of the highlights include the Senso-ji Temple, Shibuya Crossing (the world's busiest pedestrian crossing), the Imperial Palace, Meiji Shrine, Tokyo Skytree (for a great view of the endless concrete), Takeshita Street and the different parks. You should also check out Golden Gai, a small area with a lot of bars and restaurants, and the teamLab Planets museum.

\begin{wrapfigure}{L}{0.26\textwidth}
    \centering
    \vspace{-0.65cm}
    \includegraphics[width=0.25\textwidth]{figs/japan.pdf}
    \vspace{-0.25cm}
    \caption{\raggedright Sunset over Kyoto from Fushimi Inari Shrine and my \\ favorite ramen.}
    \vspace{-0.7cm}
\end{wrapfigure} 

After Tokyo, we took the local train to Fuji, where we spent one night. Unfortunately we didn't have the best weather, but we still got to see the mountain, which is a must-see when in Japan. Anyways, the area is very beautiful, and it can be nice to spend some time in the nature after a few days in the largest city in the world. Next, we took the Shinkansen to Kyoto, where we spent three nights. Kyoto is known for its temples and culture, and some of the highlights include the Fushimi Inari Shrine (the famous red gates), Bamboo forests and Gion district. We also visited Nara, a small city just outside of Kyoto, which is known for its parks with free-roaming deer. Lastly, we took the local train to Osaka, where we spent one night. Osaka is mostly known for its food and nightlife. I really recommend the Space Station bar, where you can play old video games while having a drink.

In general, I would recommend just exploring the cities, finding your own favorite spots and experiencing the culture. Regarding food, you should probably try everything (as it's all great), such as okonomiyaki (a savory pancake), takoyaki (octopus balls), sushi, nigiri, ramen and wagyu beef (we tried an A5 wagyu beef in Osaka, which was expensive, but probably one of my best food experiences ever).

\begin{figure}[H]
    \centering
    \includegraphics[width=0.99\textwidth]{figs/japan2.pdf}
    \caption{\raggedright Shinjuku Gyoen National Garden in Tokyo, Nara Park, okonomiyaki in Kyoto and A5-ranked wagyu beef in Osaka.}
\end{figure}
\vspace{-0.6cm}

\begin{wrapfigure}{r}{0.3\textwidth}
    \centering
    \vspace{-0.08cm}
    \includegraphics[width=0.29\textwidth]{figs/seoul.pdf}
    \vspace{-0.1cm}
    \caption{\raggedright Gyengbokgung Palace in Seoul.}
    \vspace{-0.35cm}
\end{wrapfigure}
\textbf{South Korea:} After Japan, I flew from Osaka to Seoul in South Korea, where I spent four nights. Here, some of the highlights include the Gyengbokgung Palace, Bukchon and Ikseon Hanok Villages, Seoul Tower (try to get there at sunset), Cheonggyecheon Stream (very nice at night when they turn on the lights by the stream) and the different markets (especially the Myengdong and Gwangjang night markets). I also visited the DMZ, a strip of land that separates North and South Korea. This was a very unique experience, and I would recommend to visit the DMZ if you have the chance. I booked the tour via \href{https://www.getyourguide.se/korean-demilitarized-zone-l89058/?visitor-id=CC8689D2A4F84D8A809D8C526DAF718E&locale_autoredirect_optout=true}{GetYourGuide}, which worked well. Regarding transportation, the subway system in Seoul is very good and efficient. The best option for this is probably to get a T-money card at any convenience store, which you can top up using cash. Moreover, Google Maps works poorly in South Korea, and most locals use \href{https://www.google.com/url?sa=t&source=web&rct=j&opi=89978449&url=https://apps.apple.com/us/app/kakaomap-korea-no-1-map/id304608425&ved=2ahUKEwj33J_Jm86NAxV5ExAIHUB2NpoQFnoECCMQAQ&usg=AOvVaw2_J4lhArArkKe5YWiU-7Jk}{KakaoMap} instead. Lastly, the food here is great, and I would recommend to try all the local dishes, like Korean BBQ, bibimbap, gimbap (Korean sushi), buchimgae (korean pancakes) and tteokbokki.

\begin{wrapfigure}{r}{0.3\textwidth}
    \centering
    \vspace{-0.53cm}
    \includegraphics[width=0.29\textwidth]{figs/taiwan.pdf}
    \vspace{-0.1cm}
    \caption{\raggedright The view from Elephant Mountain in Taipei.}
    \vspace{-0.35cm}
\end{wrapfigure}
\textbf{Taiwan:} My last stop before going back to Sweden was Taiwan, where I spent five nights in Taipei. 
Here, you should definitely include the Taipei 101 skyscraper (try to get there at sunset), some night markets (Ninhxia and Raohe night markets were great), Chiang Kai-shek Memorial Hall, Longshan Temple, Elephant Mountain (for a great view of the city) and Maokong district (known for its tea culture and nature) in your itinerary. I also spent a day in Jiufen, a small town in the mountains with a lot of history (the old street is probably the most famous attraction here), culture (try the local tea!) and shopping. Also, you should try the different food options -- dumplings, stinky tofu (if you dare, they don't taste that bad in my opinion) and bubble tea. Furthermore, the public transport system in Taipei is very good, and you can easily get around using the subway. I would recommend to get an EasyCard at one of the subway stations, which works similarly to the T-money card in Seoul.

\hrulefill

Apart from the countries above, there's of course a lot of other places to visit in Asia. From what I've heard, the Philippines is a great place, as Angkor Wat in Cambodia and Sri Lanka. China is also definitely worth a visit, however as of now, Swedes have to apply for a VISA and I believe that this process isn't the smoothest. Hopefully, this will change for future students. Lastly, Australia and New Zealand are definitely worth a visit, and not too far away from Singapore. However, then you'll have to spend a more time (and money), which I personally didn't have time for.