\chapter*{Coursework}
\addcontentsline{toc}{chapter}{Coursework}
\vspace{-0.15cm}
Generally, I believe that NTU offered a wide range of interesting courses. So, my first tip here is to consider courses outside of your major's subject (i.e. ``vidja vyerna''), e.g., from Nanyang Business School (NBS), National Institute of Education (NIE, for sports modules) or others depending on your interests. This especially applies to students from MPPHS, where the mandatory courses were done in LP1. As for me, I was happy with my course selection. I believe that I found the perfect fraction between amount of workload and interesting courses (although I had to drop some due to scheduling conflicts).

Regarding the optimal schedule, most exchangers tried to minimize the amount of classes close to the weekends due to traveling purposes. However, I had at least one lecture every day, but the courses on Thursdays and Fridays often didn't require attendance. I do believe that the attendance policy differed much between the different colleges, where courses from, e.g., NBS often had mandatory attendance and courses from SPMS (School of Physical and Mathematical Sciences) often didn't. Nevertheless, keep in mind that if you have classes on, say Thursdays or Fridays, there might be mandatory assessments on these days. I experienced this in PH4418, where I had a test every 3rd week on Thursdays. So, a tip here is to ask the lecturer if the classes are mandatory, before adding a course to your schedule. From my experience, most lecturers answered quickly and were very helpful. Also, I'd recommend minimizing the number of exams. That way, you'll have more continuous assessment, and thus more time to travel when school ends. So, in short, I'd say that an optimal schedule for a travel-interested student was to have classes three days a week (which was definitely possible), e.g., on Mondays, Tuesdays, and Wednesdays, while minimizing the number of exams and focusing more on continuous assessment.

Finally, as I was the first MSc student from MPPHS at NTU ever (or at least in a long time), I hadn't much to go on regarding relevant courses for my program. However, \href{https://www.ntu.edu.sg/spms}{the CoS website} provided useful information about the courses available, so this was never an issue. Furthermore, the accreditation was made such that 4 AU = 7.5 hp, and NTU recommended students to study between 15 -- 20 AUs per semester (I did 18). In practice, the accreditation was done by logging on to the \href{https://mex.portal.chalmers.se}{Chalmers MEX Portal} and create a request for accreditation of your attended courses. Here, you had to attach the syllabus for each course along with your ToR from NTU. For the latter, I used the \href{https://wis.ntu.edu.sg/webexe/owa/request_transcript_und.LoginN?pType=SH}{Self-help Transcript System} to generate a non-signed version of this when I'd gotten my grades in the beginning of June (I had to meet with my DoS in person to show how I retrieved this since it wasn't signed). If you're more patient than me though, NTU sent your coordinator (and possibly also your DoS) a signed version that arrived in the mid or end of July. Anyway, at last your request was reviewed (and hopefully approved) by your DoS, see \href{https://www.chalmers.se/en/education/your-studies/plan-and-conduct-your-studies/transferring-credits/}{this page} for more information.
\section*{Course Descriptions and Impressions}
\phantomsection
\addcontentsline{toc}{section}{Course Descriptions and Impressions}
Below, I've listed the courses I attended during my exchange, along with their syllabus descriptions and my personal impressions.

\vspace{-0.3cm}
\hrulefill

{\large{\textbf{Open Quantum Systems $\vert$ PH3406} \hfill \textbf{4 AU}}}

\vspace{-0.3cm}
\textbf{Accredited as: MCC180 (compulsory elective) 7.5 hp \hfill Exam: Yes}

{\large{\textsc{Lecturer/Examiner:} \textit{Dr. Koh Teck Seng}}} \vspace{0.25cm}

{\large{\textbf{\textit{Content}}}} 
\vspace{-0.25cm}

This course covers key concepts, including density matrices, the interaction picture, the Born-Markov and Lindblad master equations, decoherence, relaxation, dephasing, and linear dynamical maps. It explores canonical models (collisional, spin-boson, spin-spin) and connects quantum error correction to the Kraus representation theorem.

{\large{\textbf{\textit{Impression}}}}
\vspace{-0.25cm}

I really enjoyed this course, as it was a good mix between theory and practical applications. Dr. Koh was very nice, a great lecturer, and very passionate about the subject. The course was very well organized and assessed through one homework assignment, one project report, one midterm, and one final exam. The homework and project were great for understanding the material and involved coding -- aligning the course with quantum computing, which I found very interesting. The midterm and exam were quite hard, but I only strived for a pass, so the amount of extra work I put in was more from my own interest than from necessity. I'd recommend this course to anyone interested in quantum mechanics, as it was a great introduction to the open quantum systems.

\vspace{-0.3cm}
\hrulefill

{\large{\textbf{Physics in the Industry $\vert$ PH4418} \hfill \textbf{4 AU}}}

\vspace{-0.3cm}
\textbf{Accredited as: Elective 7.5 hp \hfill Exam: No}

{\large{\textsc{Lecturer/Examiner:} \textit{Prof. Lew Wen Siang \& Dr. Leek Meng Lee}}} \vspace{0.25cm}

{\large{\textbf{\textit{Content}}}} 
\vspace{-0.25cm}

This course explores major industries in Singapore where physics plays a key role. It provides an overview of diverse applications, helping students understand career opportunities and make informed employment decisions after graduation.

{\large{\textbf{\textit{Impression}}}}
\vspace{-0.25cm}

This course was divided into four topics: semiconductors, photonics, food- and biomedical physics. Each topic was taught by a different lecturer, and the assessments consisted of in-class tests after each topic, and a final report and presentation (which could be done in pairs). The tests were rather easy, while the latter demanded some extra work. However, the topics and lecturers didn't really catch my interest and I'd probably only recommend this as a filler course. Nevertheless, if you find the topics more interesting than I did, you might learn a lot and possibly also discover a future career path.

\vspace{-0.4cm}
\hrulefill

\newpage

\vspace{-0.3cm}
\hrulefill

{\large{\textbf{Leadership in the 21st Century $\vert$ BU5642} \hfill \textbf{3 AU}}}

\vspace{-0.3cm}
\textbf{Accredited as: MTS-Course with HY0001 (elective) 5.6 hp \hfill Exam: No}

{\large{\textsc{Lecturer/Examiner:} \textit{Assoc. Prof. Jing Zhu}}} \vspace{0.25cm}

{\large{\textbf{\textit{Content}}}} 
\vspace{-0.25cm}

The Leadership in the 21st Century course takes an evidence-based approach to leadership theory, frontier research, and real-world applications. Through interactive seminars with role-plays, case studies, and discussions, students develop leadership skills for managerial, consulting, or personal growth.

{\large{\textbf{\textit{Impression}}}}
\vspace{-0.25cm}

Together with HY0001, this course provided me with MTS-credits at Chalmers, which was very convenient for me. However, in contrary to HY0001, this course actually contained valuable content (in my opinion). The lecturer, Jing, was very driven, kind and wanted everybody to succeed and gain something from the course. I personally believe that everyone would benefit from this course, even those that don't aspire to become leaders in the future. Much of the content could be boiled down to ``knowing and understanding people'', often in a setting where leadership played a role. If you put your whole heart and effort into this course, I believe that you'll grow much, both as a person and as a leader, and you'll be well-equipped for your future career. The assessments consisted of two quizzes and a team project with a written report and presentation, and I wouldn't say that the workload was heavy by any means.

\vspace{-0.3cm}
\hrulefill

{\large{\textbf{Ethics and Moral Reasoning $\vert$ HY0001} \hfill \textbf{1 AU}}}

\vspace{-0.3cm}
\textbf{Accredited as: MTS-Course with BU5642 (elective) 1.9 hp \hfill Exam: No}

{\large{\textsc{Lecturer/Examiner:} \textit{Jacob Mok}}} \vspace{0.25cm}

{\large{\textbf{\textit{Content}}}} 
\vspace{-0.25cm}

This course introduces key moral values like benevolence, impartiality, and integrity through major ethical theories. It fosters critical thinking on complex moral issues, explores academic integrity and research ethics, and examines the role of ethics in environmental sustainability.

{\large{\textbf{\textit{Impression}}}}
\vspace{-0.25cm}

In this course, all content was delivered online -- even the assessments. You had to go through a number of lectures, each ending with a quiz. The quizzes were very easy, as you could do them up to three times using all available resources. Except for these, you also had to write a short text on one of the topics, and a peer-review on another student's text. From my experience, most of the content in the course fell into the ``common sense'' category, and I didn't learn much from the course. However, it was a solid filler course, and most importantly, it gave me MTS-credits at Chalmers, which was great for me since my Bachelor's didn't provide me with these. So, if you need MTS-credits, this course is a great option together with BU5642, but otherwise I wouldn't recommend it.

\vspace{-0.4cm}
\hrulefill

\newpage

\vspace{-0.3cm}
\hrulefill

{\large{\textbf{C \& C++ Programming $\vert$ SC1008} \hfill \textbf{3 AU}}}

\vspace{-0.3cm}
\textbf{Accredited as: Elective 5.6 hp \hfill Exam: No}

{\large{\textsc{Lecturer/Examiner:} \textit{Assoc. Prof. Hui Siu Cheung}}} \vspace{0.25cm}

{\large{\textbf{\textit{Content}}}} 
\vspace{-0.25cm}

This course introduces foundational concepts in C and C++, focusing on system programming, embedded systems, and performance optimization. It covers applications in gaming engines, virtual reality, web browsers, databases, and blockchain technology.

{\large{\textbf{\textit{Impression}}}}
\vspace{-0.25cm}

In this programming course, the lectures were conducted online, and there was also a mandatory lab/tutorial session each week. The course was divided into one C part and one C++ part, which contained assessed assignments that were due every other week (all tools were allowed for these). Apart from these assignments, the course was assessed through a final test at the end of each part (no tools were allowed for these). For the C part, this contained both coding and multiple choice questions (MCQs), and for the C++ part, only MCQs. I'd probably only recommend this course for two reasons -- either if you like programming and want to add to your repertoar, or as a filler course, as the level was basic and the organization of the course fit the exchanger life very well.

\vspace{-0.3cm}
\hrulefill

{\large{\textbf{Rugby $\vert$ SS5205} \hfill \textbf{3 AU}}}

\vspace{-0.3cm}
\textbf{Accredited as: Elective 5.6 hp \hfill Exam: No}

{\large{\textsc{Lecturer/Examiner:} \textit{Harrie Desianto Hussien}}} \vspace{0.25cm}

{\large{\textbf{\textit{Content}}}} 
\vspace{-0.25cm}

This rugby course teaches fundamental skills, strategies, and game principles through tag and touch rugby. It combines theory and practice using the Sport Education and Games Concept Approach (GCA) to enhance understanding and gameplay competency.

{\large{\textbf{\textit{Impression}}}}
\vspace{-0.25cm}

This course was definitely one of my favorites during the semester. And before you reject it due to the physicality of regular rugby, you should know that this was a non-contact course (touch rugby). This changes the game a lot, as you don't have to worry about receiving brutal tackles or getting injured (I kind of looked forward to the physical aspects though). However, it was probably for the best to avoid injuries while abroad, and I still believe that you'll learn a lot about rugby as a sport, as contact- and touch rugby are similar in many ways. We met once a week for three hours, and the course was assessed through an easy theoretical quiz and a practical assessment that took place during the last two classes. In the beginning, the focus was more on ball handling, rules and tactics. Then, during the last four or five weeks, each lesson essentially contained only gameplay with short breaks in between. Furthermore, Coach Harrie was very nice, experienced and passionate about the sport. I'd definitely recommend this course to anyone interested in sports, especially while on exchange, as it was fun, social, and had a minimal workload.

\vspace{-0.4cm}
\hrulefill

\newpage

\section*{Course Rankings}
\phantomsection
\addcontentsline{toc}{section}{Course Rankings}
Here, I've summarized my impressions of the courses I attended and ranked them by workload, content, and teaching quality.

\begin{center}
    \hrulefill {\Large\textsc{ Open Quantum Systems}} \hrulefill
    \vspace{0.5cm}
\renewcommand{\arraystretch}{1.5}
\begin{tabular}{>{\centering\arraybackslash}p{0.23\textwidth} >{\centering\arraybackslash}p{0.23\textwidth} >{\centering\arraybackslash}p{0.23\textwidth} >{\centering\arraybackslash}p{0.23\textwidth}}
    \large{\textbf{Workload}} & \large{\textbf{Content}} & \large{\textbf{Teaching}} & \large{\textbf{Total}} \\
    \ratingsquare{3}{5} & \rating{5}{5} & \rating{5}{5} & \rating{5}{5} \\ 
\end{tabular}
\end{center}
\vspace{-0.5cm}
\hrulefill

\vspace{0.6cm}

\begin{center}
    \hrulefill {\Large\textsc{ Physics in the Industry}} \hrulefill
    \vspace{0.5cm}
\renewcommand{\arraystretch}{1.5}
\begin{tabular}{>{\centering\arraybackslash}p{0.23\textwidth} >{\centering\arraybackslash}p{0.23\textwidth} >{\centering\arraybackslash}p{0.23\textwidth} >{\centering\arraybackslash}p{0.23\textwidth}}
    \large{\textbf{Workload}} & \large{\textbf{Content}} & \large{\textbf{Teaching}} & \large{\textbf{Total}} \\
    \ratingsquare{2}{5} & \rating{2}{5} & \rating{2}{5} & \rating{2}{5} \\ 
\end{tabular}
\end{center}
\vspace{-0.5cm}
\hrulefill

\vspace{0.6cm}

\begin{center}
    \hrulefill {\Large\textsc{ Leadership in the 21st Century}} \hrulefill
    \vspace{0.5cm}
\renewcommand{\arraystretch}{1.5}
\begin{tabular}{>{\centering\arraybackslash}p{0.23\textwidth} >{\centering\arraybackslash}p{0.23\textwidth} >{\centering\arraybackslash}p{0.23\textwidth} >{\centering\arraybackslash}p{0.23\textwidth}}
    \large{\textbf{Workload}} & \large{\textbf{Content}} & \large{\textbf{Teaching}} & \large{\textbf{Total}} \\
    \ratingsquare{2}{5} & \rating{4}{5} & \rating{5}{5} & \rating{4}{5} \\ 
\end{tabular}
\end{center}
\vspace{-0.5cm}
\hrulefill

\vspace{0.6cm}

\begin{center}
    \hrulefill {\Large\textsc{ Ethics and Moral Reasoning}} \hrulefill
    \vspace{0.5cm}
\renewcommand{\arraystretch}{1.5}
\begin{tabular}{>{\centering\arraybackslash}p{0.23\textwidth} >{\centering\arraybackslash}p{0.23\textwidth} >{\centering\arraybackslash}p{0.23\textwidth} >{\centering\arraybackslash}p{0.23\textwidth}}
    \large{\textbf{Workload}} & \large{\textbf{Content}} & \large{\textbf{Teaching}} & \large{\textbf{Total}} \\
    \ratingsquare{0}{5} & \rating{2}{5} & \rating{1}{5} & \rating{2}{5} \\ 
\end{tabular}
\end{center}
\vspace{-0.5cm}
\hrulefill

\vspace{0.6cm}

\begin{center}
    \hrulefill {\Large\textsc{ C \& C++ Programming}} \hrulefill
    \vspace{0.5cm}
\renewcommand{\arraystretch}{1.5}
\begin{tabular}{>{\centering\arraybackslash}p{0.23\textwidth} >{\centering\arraybackslash}p{0.23\textwidth} >{\centering\arraybackslash}p{0.23\textwidth} >{\centering\arraybackslash}p{0.23\textwidth}}
    \large{\textbf{Workload}} & \large{\textbf{Content}} & \large{\textbf{Teaching}} & \large{\textbf{Total}} \\
    \ratingsquare{2}{5} & \rating{2}{5} & \rating{2}{5} & \rating{2}{5} \\ 
\end{tabular}
\end{center}
\vspace{-0.5cm}
\hrulefill

\vspace{0.6cm}

\begin{center}
    \hrulefill {\Large\textsc{ Rugby}} \hrulefill
    \vspace{0.5cm}
\renewcommand{\arraystretch}{1.5}
\begin{tabular}{>{\centering\arraybackslash}p{0.23\textwidth} >{\centering\arraybackslash}p{0.23\textwidth} >{\centering\arraybackslash}p{0.23\textwidth} >{\centering\arraybackslash}p{0.23\textwidth}}
    \large{\textbf{Workload}} & \large{\textbf{Content}} & \large{\textbf{Teaching}} & \large{\textbf{Total}} \\
    \ratingsquare{1}{5} & \rating{4}{5} & \rating{5}{5} & \rating{5}{5} \\ 
\end{tabular}
\end{center}
\vspace{-0.5cm}
\hrulefill
