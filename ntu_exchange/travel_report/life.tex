\chapter*{Life at NTU}
\phantomsection
\addcontentsline{toc}{chapter}{Life at NTU}
Living at NTU and in Singapore was a great experience (and adventure), and I'd say that it was a perfect place for an exchange. Campus was huge and beautiful, with lots of nature built into it, and the city was very clean and safe. Moreover, the locals were friendly and the food was amazing. Also, NTU -- like other universities in Singapore -- accepted \textit{lots of} exchangers, which enabled you to not only experience and learn about the culture in Singapore, but cultures from all over the world. I met people from virtually every corner of the globe. However, there were things that you might've wanted to know before leaving Sweden, which I'll try to cover in this chapter. Hopefully, it'll decrease the culture shock (if you get one) and make you more prepared for your studies at NTU.
\vspace{-0.2cm}
\section*{Everyday Life}\label{everyday}
\phantomsection
\addcontentsline{toc}{section}{Everyday Life}
Despite being on the other side of the globe, you'll sooner or later find your personal routine at NTU. For me, I believe that I found mine after some weeks or so, but I'm sure that this varies depending on how used you are to change. A general tip is to find friends (either locals or other exchangers) so that you can navigate the new surroundings together, which will be more fun and probably easier. Of course, many things weren't that different from your usual life back home. Nevertheless, there were things that were substantially different from those in Sweden.

For starters, Singapore was \textit{hot}. The temperature during day was around 30 -- 35$^\circ$C, and the humidity was very high. The sun was also very strong, so sunscreen and water were your new best friends. On the contrary, the evenings were nice with a temperature around 25$^\circ$C, and you'll notice that the locals generally are more active at night, likely due to the climate. Eventually, you'll probably get somewhat used to the heat -- not like the locals though, who considered 28$^\circ$C ``cold''. Moreover, the weather was also very unpredictable, and the rain could be very heavy, but usually didn't last for long. For this reason, I'd recommend always having an umbrella close at hand. To escape the heat, you'll probably spend a lot of time in the AC. At NTU, many canteens along with most study rooms and libraries were air-conditioned, which initially made these my favorite spots on campus.

Secondly, the cooking culture was completely different from Sweden. As I'll also explain in the \hyperref[discover]{final chapter}, a large part of the food culture in Singapore was based on the so-called hawker centers (or canteens at NTU). These were basically large food courts, where you could find dishes from all over the world. This made it cheap and easy to eat out, and I'd say that most students at NTU took advantage of this. 

In fact, the facilities for cooking your own food were very limited at NTU, and I'd say that it wasn't worth the effort to cook. However, if you were a fan of cooking, you could use the pantry in your hall, which (in most cases) was equipped with some hotplates. However, this was where the canteens at NTU entered the picture, and I'd recommend trying all of them and finding your own favorites (the same goes for the dishes). Generally, I'd say that food was cheaper in hall canteens, while the larger ones were more expensive but offered a wider range of dishes. Below, I've listed my favorite canteens at NTU:
\vspace{-0.3cm}
\begin{itemize}
    \item[--] \textbf{Canteen 2}: Strategically placed in the middle of campus with (in my opinion) some of the best meals. It was also cheaper than the North and South Spine canteens, with the downside that no stall accepted cashless payments.
    \item[--] \textbf{North Hill Canteen}: Here, you could find a wide range of nice dishes and it was usually not too crowded. However, here you had to rely a bit more on cash.
    \item[--] \textbf{North and South Spine Canteens}: The largest canteens on campus, which were usually near your lecture halls. They had a wide range of dishes, but were often crowded and a bit more expensive. Most stalls here accepted cashless payments.
    \item[--] \textit{Honorable Mention}: The Indian food at my canteen (Tamarind Hall). This explains itself, probably the best Indian food on campus (and my favorite dish).
\end{itemize}
\vspace{-0.3cm}
\begin{figure}[H]
    \centering
    \includegraphics[width=0.99\textwidth]{figs/food.pdf}
    \caption{\raggedright Bottom right: Canteen 2 -- one of my favorite canteens. Center was my go-to meal on campus, Chinese La-mian noodles and chicken. Soups (in all forms) were also common to see. Bottom left was another personal favorite which seemed weird at first -- chicken omelette rice. Lastly, top right was my favorite dish, Indian food at Tamarind Hall.}
\end{figure}
\vspace{-0.6cm}
Lastly, since campus functioned much like a small town, there were some grocery stores and shops where you could buy essentials, e.g., near Canteen 2 and in North Spine. For accessing larger malls and grocery stores, you could take the campus bus to Pioneer MRT station and go to Jurong East. However, a closer option was to take the 199 bus from campus, which took you to Jurong Point -- a large mall with lots of stores and food stalls (here, you could also access the MRT at Boon Lay station).
\section*{Housing}
\phantomsection
\label{house}
\addcontentsline{toc}{section}{Housing}
\vspace{-0.2cm}
After I was accepted to NTU, I applied for campus housing in the second half of November. Regarding living on or off campus, I'd recommend taking a peek at some of the previous travel reports (e.g., that from Filip in 2022/23). However, in short I can say that I was 110 \% happy with my decision to live on-campus. In my opinion, these were the main pros and cons with each option:
\begin{table}[H]
    \begin{center}
        \begin{tabular}{||c c | c c||}
            \toprule
            \multicolumn{2}{||c|}{\textbf{Living on Campus}} & \multicolumn{2}{c||}{\textbf{Living off Campus}} \\ \toprule
            \textsc{Pros} & \textsc{Cons} & \textsc{Pros} & \textsc{Cons} \\ \hline\hline
            Cheaper & \makecell{Room \& facility \\ standard lower} & \makecell{Better facilities \\ (not guaranteed)} & More expensive \\ \midrule
            Closer to NTU & Far from downtown & Closer to downtown & Far from NTU \\ \midrule
            \makecell{Most exchange \\ students lived here} & -- & -- & \makecell{More complicated} \\ \midrule
            \makecell{Engage more in \\ the NTU life} & -- & -- & -- \\ \bottomrule
        \end{tabular}
    \end{center}
\end{table}
\vspace{-1cm}

As the table suggests, I believe that you'd meet more exchangers and soak up more of the NTU student life (e.g., eating in the canteens and engaging in hall activities) living on campus. Of course, room and facility standards were most likely better off campus, but at a higher price. Moreover, I'd say that the procedure for securing off campus accommodation was generally more complicated than on campus. Lastly, before sharing my own housing experience, I'd like to mention that I didn't hear of a single person who applied for campus housing and didn't get it (which I know can be a concern).

\begin{wrapfigure}{r}{0.31\textwidth}
    \centering
    \vspace{-0.67cm}
    \includegraphics[width=0.27\textwidth]{figs/room.pdf}
    \vspace{-0.4cm}
    \caption{My room at Tamarind Hall.}
    \vspace{-0.4cm}
\end{wrapfigure}
Now, I applied for a single room with AC and was assigned this at Tamarind Hall (or Tama) in the north-west of campus. I heard some weird stories from my friends who applied for a single room and were assigned a double and vice versa, but this must've been rare. Anyway, I was satisfied with my dorm and hall, which was one of the newest at NTU. For the whole semester, I paid around 20\,000 SEK, which wasn't cheap considering the standard, but much cheaper than living off campus (also, the AC cost around 0.4 SGD per hour of use). From my experience, the standard was more or less comparable all over campus. You probably saw lizards, bugs and cockroaches independently of where you lived, and eventually you got somewhat used to it. The shared bathrooms weren't exactly first class either, but I never had any trouble using them. In my hall (and I think that this applied to most of them), we also had access to a small pantry with cold and hot water, some hotplates and a microwave. There were also washing and drying machines on certain floors, where each program cost 1 SGD coin (this made these coins extremely precious, so make sure to not waste them). For the latter, you could either ask to get some extra 1ers when buying food, if you payed a little extra. Or, if you had a lot of small coins laying around, you could use the vending machines to exchange these for 1ers.
\begin{figure}[H]
    \centering
    \includegraphics[width=0.99\textwidth]{figs/fac1.pdf}
    \vspace{-0.15cm}
    \caption{\raggedright The shared bathroom and pantry on my floor.}
\end{figure}
\vspace{-0.6cm}
Something I haven't seen in previous travel reports (maybe for obvious reasons) is the problem with mould in the halls. This was a common problem in Singapore due to the high humidity, and I along with many others (mostly guys, actually) experienced issues with this during the first months. For example, some of my clothes (!) had white mould after I got home from a week of traveling. I reported this to my hall and they sent two women to ``educate'' me on the Singaporean climate. However, their tips actually seemed to work, as I didn't get much more mould after that. Their advice was to keep the ceiling fan off while keeping the window and ventilation open when being away for a longer period of time. As for the clothes, I used vinegar and hot water to clean them, and then I washed them using regular detergent along with some vinegar in the washing machine (this, washing your clothes often, also seemed like an effective precaution against mould). You could also buy a dehumidifier (I didn't though), which should lower the humidity in your room. I know that some students bought these at North Spine, but you could probably find them in the malls at Jurong East and Jurong Point as well.
\vspace{-0.08cm}
\section*{Economy}
\phantomsection
\label{eco}
\addcontentsline{toc}{section}{Economy}
\begin{wrapfigure}{l}{0.31\textwidth}
    \centering
    \vspace{-0.57cm}
    \includegraphics[width=0.3\textwidth]{figs/economy.pdf}
    \vspace{-0.1cm}
    \caption{\raggedright The large financial district in downtown Singapore.}
    \vspace{-0.8cm}
\end{wrapfigure}
Relative to other Southeast Asian countries, Singapore was expensive. Clothes, electronics, and groceries were approximately at the same price level as in Sweden, if not higher in some cases. Alcohol was very expensive at bars and clubs, but there were areas downtown (e.g., Boat Quay, similar to 2a lång) where you could buy a beer for around 50 SEK. When it came to food, you already know about the hawker centers, where you could get a proper meal for around 30 -- 50 SEK. Only NTU had 10 -- 15 of these.

Nevertheless, I'd argue that the greatest concern -- and difference from Sweden -- was the cash situation. This was mostly an issue at hawker centers, where many food stalls didn't accept card payment. As mentioned, once you complete your STP, you'll receive your matriculation card, which had a NETS functionality (a local payment method). This could be topped up through an app (disclaimer: I didn't try this), and used in many places where card payments weren't accepted, which made it an alternative to cash.

However, I went for an alternative solution, which I believe was the best one -- namely opening a Singaporean bank account, which I'll explain below. This way, you didn't have to worry about cash, or your matriculation card and you could pay directly from your phone. Before opening my bank account, I solely relied on cash and my \href{https://wise.com}{Wise} card. This worked smoothly, however if you're as allergic to cash as I am, I'd recommend that you get a bank account ASAP. Regarding using Wise or \href{https://www.revolut.com/sv-SE/}{Revolut} as your travel card (which seemed to be the two options that offered the best exchange rates and lowest fees), I had a great experience using Wise. However, there were pros and cons with both (I found more pros with Wise though) and I know that most Swedish exchangers used Revolut.
\subsection*{Opening a Singaporean Bank Account}
\phantomsection
\addcontentsline{toc}{subsection}{Opening a Singaporean Bank Account}
Once you got your STP, you could apply for Singpass \href{https://portal.singpass.gov.sg/home/ui/register/instructions}{here}. This was basically the BankID of Singapore, and after approval, you could reach this through the \href{https://www.google.com/url?sa=t&source=web&rct=j&opi=89978449&url=https://apps.apple.com/sg/app/singpass/id1340660807&ved=2ahUKEwj8ouKcoaWLAxVyUGcHHd3CKWgQFnoECBUQAQ&usg=AOvVaw2MU54ywzShBUAGdsGy2E40}{Singpass app}. Then, you could easily apply for a bank account at \href{https://www.dbs.com.sg/index/default.page}{DBS}, one of the major banks in Singapore (there were other options, but I can't say much about their application processes). This was done in the \href{https://www.google.com/url?sa=t&source=web&rct=j&opi=89978449&url=https://apps.apple.com/sg/app/dbs-digibank/id1068403826&ved=2ahUKEwjWjsikoqWLAxW2UGwGHWeHHRsQFnoECBcQAQ&usg=AOvVaw3MmyWfIWdIEfDqXsxZtKcG}{DBS digibank app}, where you later also managed your balance, made payments, etc.. Apart from Singpass, you needed proof of your residential address in Singapore, which you could get from your hall's office. You also had to provide proof of your Swedish tax liability (I used an English Personbevis from Skatteverket, which could be retrieved \href{https://www.skatteverket.se/privat/etjansterochblanketter/allaetjanster/tjanster/skrivutpersonbevis.4.18e1b10334ebe8bc80001262.html}{here} after login, by choosing \textit{``Utdrag om folkbokföringsuppgifter -- engelsk text''}). Then, your application was processed for a few days before you got a confirmation email from DBS. They also sent a debit card to your address. 

This made life much easier, as you also gained access to the Singapore equivalent of Swish -- PayNow. Similarly to the NETS functionality of the matriculation card, PayNow worked through QR codes accepted almost everywhere in Singapore. Another bonus of getting a bank account in Singapore was that if you did need cash, the withdrawals were free, unlike if you used your Swedish card (of course, they weren't totally free, as you paid some exchange fees when sending money from Sweden to your DBS account). DBS also offered some great features for travelers, e.g., allowing account openings in different currencies and fee-free transfers between them (similar to Wise and Revolut). Lastly, to transfer money from my Swedish account to my Singaporean one, I used Wise (Revolut was also an option), which had the best exchange rates and lowest fees.

As a side note, you might experience issues trying to type decimal numbers when using PayNow. In this case, change your phone's decimal format to ``.'' from ``,''. On an iPhone, this was done under General settings $\rightarrow$ Language \& Region $\rightarrow$ Number Format. % Allmänt $\rightarrow$ Språk och region $\rightarrow$ Numeriskt format.
\section*{Education -- Chalmers vs. NTU}
\phantomsection
\addcontentsline{toc}{section}{Education -- Chalmers vs. NTU}
Despite being almost 10\,000 km apart, Chalmers and NTU shared a lot when it came to the studying experience. All in all, you'll be a university student regardless of where you wake up in the mornings, and once you hit your daily routines, you'll probably almost forget that you're on the other side of the world. However, as soon as you escape the classroom or your dorm (and the AC), there'll definitely be some differences.

To begin with, the NTU campus was huge (and beautiful), and the surroundings made you feel like you were in the middle of a Jurassic Park movie, with large trees and plants. In the beginning of your exchange, you'll definitely be overwhelmed by its size and use Google Maps \textit{a lot}. But as time goes on, you'll find that it isn't too large and that you often can get around by foot. Although walking was possible, the heat (or rain) often made \href{https://www.google.com/url?sa=t&source=web&rct=j&opi=89978449&url=https://apps.apple.com/us/app/ntu-omnibus/id1636457987&ved=2ahUKEwiV0aGN46-KAxWHIxAIHXUHBg0QFnoECBsQAQ&usg=AOvVaw0nbsveDMT3B2KE9kmpcAZ_}{the free buses}, that ran frequently throughout the day (less frequent in the evenings) a better alternative. Despite the availability of buses, they tended to be very crowded during school hours, making walking a viable alternative if the distance wasn't too long.
\vspace{-0.25cm}
\begin{figure}[H]
    \centering
    \includegraphics[width=0.99\textwidth]{figs/environment.pdf}
    \caption{\raggedright The campus voted as one of the most beautiful in the world. Clockwise from the top left: Gaia (NBS building), The Hive, Yunnan Garden, the view from my resident hall and the walkway from North Spine to my hall.}
\end{figure}
\vspace{-0.6cm}
A small disclaimer is that Singapore's new MRT line, Jurong Region Line, is under construction. This made much of the campus look like a construction site, although it didn't affect everyday life much. The good news is that future students won't be as dependent on the MRT station at Pioneer (which was only accessible by bus at the time of my exchange) in order to go downtown, as the construction is planned to be finished in 2029. 

Academically, NTU emphasized continuous assessment, with midterms and quizzes playing a significant role in most courses. Unlike at Chalmers, where finals often constitute a large percentage of the final grade, NTU's exams rarely accounted for more than 60 \% of the grade. This structure encouraged students to engage with the material consistently throughout the semester, which I actually believe was a better way of learning. Another key difference, which I wasn't as happy with, was the academic calendar. At NTU, all courses ran concurrently throughout the semester, without the distinct reading periods used at Chalmers. This structure required students to manage multiple subjects simultaneously over a long period, which I generally disliked. Regarding academics, the quality of teaching, course content, and overall level of difficulty were generally good. However, given NTU's high global rankings, I expected an even higher level. In comparison, I found that many courses I've attended at Chalmers were more challenging in terms of depth and difficulty -- though, of course, this depends on the total effort you put in.

\begin{figure}[H]
    \centering
    \includegraphics[width=0.99\textwidth]{figs/study_places.pdf}
    \caption{\raggedright Great study facilities: Lee Wee Nam Library and outside at North Spine Plaza.}
\end{figure}
\vspace{-0.6cm}
Moreover, the grading system at NTU differed from that at Chalmers. The highest grade was A+, while the lowest was F, and grades were generally assigned based on a bell curve (I'm not entirely sure that this was applied to exchangers though). This meant that students weren't only assessed on their individual performance but also in comparison to their peers. However, as Chalmers students in general can't transfer grades from NTU to Chalmers, there was essentially no need to aim for the highest grade (instead I believe that you should focus on traveling, exploring new cultures, and personal development).

Lastly, the student culture at NTU was also notably different from that at Chalmers. There was a strong academic focus, with almost no partying. The student-led clubs were almost always of more serious nature, often focused on academic or professional activities rather than the ``sexmästerier'' common at Chalmers. A unique aspect of student life was the hall culture. Each residential hall had its own traditions, events, and even sports teams. One of the most notable traditions is the Hall Olympiad, held in autumn, where residents of the different halls compete against each other in various sports.
\section*{Free Time and Activities}
\phantomsection
\addcontentsline{toc}{section}{Free Time and Activities}
\begin{wrapfigure}{l}{0.45\textwidth}
    \centering
    \vspace{-0.53cm}
    \includegraphics[width=0.44\textwidth]{figs/gym.pdf}
    \vspace{-0.1cm}
    \caption{\raggedright The gym at my hall.}
    \vspace{-0.32cm}
\end{wrapfigure}
After school hours, there was a lot to do around campus. As I mentioned earlier, campus felt like a small town, and there were many activities and events to engage in. For example, there were many sports facilities, such as gyms, swimming pools, and sports courts, which were free for students. During my exchange semester, the only swimming pool open was the one at NIE, which was nice and never seemed too crowded. Concerning gyms, many halls had their own (including my hall), but these were mostly small with limited equipment. The most easily accessible large gym — and the one that I used most during the semester — was at North Hill. However, this got very crowded during the evenings (not to the extent that it became unusable though). You could also look into getting a membership at a gym outside of campus, but this would be a bit of a hassle. 

Regarding other sport facilities, you could engage in most sports at NTU. I frequently played table tennis, badminton, and floorball. For racket sports, there was a large multi-purpose hall at North Hill (this was where I played the most), where you often had to book your time (could be done \href{https://www.google.com/url?sa=t&source=web&rct=j&opi=89978449&url=https://ntu.facilitiesbooking.com/&ved=2ahUKEwi-k6Pays2LAxUg2TgGHZjtK4sQFnoECAsQAQ&usg=AOvVaw0UI5Z1OVQuEv2rAMKN2LGh}{here} after login), and for some sports, especially badminton, available times were rare. However, there were also many sports clubs at NTU, which you could join for a small fee. I joined the NTU Floorball Club for 5 SGD due to my history with the sport, and they arranged game-play two times a week. This was a great way to meet locals while also maintaining my floorball skills. 

In addition to sports, there were many other types of clubs, focusing on for example music or dance. Most halls also arranged different activities for its residents, which were often communicated via the NTU email. For example, one of my friends and I attended a session of Muay Thai at my hall. Also, there were many events for exchange students arranged by NTU in the start of the semester, which I definitely recommend in order to meet other exchangers. There were also several parties arranged for exchangers (and non-exchangers) throughout the semester (mostly in the beginning though). For this, I used the \href{https://aentry.app/events}{Aentry app}, where you could find and buy tickets to many bars and clubs. Lastly, studying in Singapore was, of course, the perfect ground zero for traveling in Asia, which I'll cover in the \hyperref[discover]{final chapter}.

\begin{figure}[H]
    \centering
    \includegraphics[width=0.99\textwidth]{figs/sports.pdf}
    \caption{\raggedright Great sports facilities: The running track at The Wave, the multi-purpose hall at North Hill with badminton courts and table tennis, the swimming pool and hockey pitch (where I played rugby) at NIE, and the volleyball court outside Canteen 2.}
\end{figure}
