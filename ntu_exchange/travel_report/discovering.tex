\chapter*{Discovering Asia and Singapore}
\phantomsection
\label{discover}
\addcontentsline{toc}{chapter}{Discovering Asia and Singapore}
\vspace{-0.15cm}
One of the best things about studying abroad, especially in Singapore, was the opportunity to travel and explore new places. As a country, Singapore didn't have that much to offer. Still, there were some great places to visit and things to explore, which I've described below. When I asked local students, they often agreed that the best thing about Singapore was its location, as it made it easy to leave the country and explore the rest of Asia. This might sound strange and a bit sad, but after almost half a year in Singapore, I must agree on this. The country is small, but the location was perfect for exploring the rest of Asia, which I'll also cover in this chapter.
\vspace{-0.15cm}
\section*{Singapore -- City and Country in One}
\phantomsection
\addcontentsline{toc}{section}{Singapore -- City and Country in One}
\vspace{-0.1cm}
Even though campus was large, beautiful and had a lot to offer, you'll eventually want to explore what's behind the doors of NTU. Singapore is a small country (and city), and you could easily explore most things in a few days. The public transport system was great, and you could get around easily. The MRT (the subway) was very cheap and easy to use, and you could also use buses to get around. I believe that you could get some sort of card for the public transport, but I'd recommend using your regular bank card (or Apple/Google Pay) to pay for the rides. This was very convenient, as you didn't have to worry about getting a card or topping it up. Of course, you could also use Grab (the Uber of Southeast Asia) to get around, which was relatively cheap and available 24/7 in contrary to the public transport (the MRT and most bus lines closed around midnight).

As for things to see and explore, there were some places that shouldn't be missed. Below, I've listed some of the most popular ones to visit sorted by category. Regarding the hawker centers, those mentioned here were probably the most popular and touristic ones, but there were essentially an infinite number of hawker centers to explore in total.

\textbf{Nature:}
\vspace{-0.3cm}
\begin{itemize}
    \item \textbf{Sentosa Island:} A small island in the south of Singapore, known for its beaches, resorts, and attractions (e.g., aquariums, shops, and Universal Studios Singapore).
    \item \textbf{Gardens by the Bay:} A park with futuristic gardens and structures. The Supertree Grove was a must-see, especially at night when the trees were lit up.
    \item \textbf{MacRitchie Reservoir:} A large nature reserve with hiking trails. Here, you could also find the famous treetop walk -- a suspension bridge with a great view.
\end{itemize}
\vspace{-0.25cm}
\textbf{Culture:}
\vspace{-0.3cm}
\begin{itemize}
    \item \textbf{Chinatown:} A famous Chinatown with lots of history and culture.
    \item \textbf{Little India:} A colorful neighborhood with lots of Indian culture. Here, you could find temples, markets, and lots of great food.
    \item \textbf{Orchard Road:} A famous shopping street with lots of malls and luxury stores.
    \item \textbf{Raffles Hotel:} A historic hotel where you could find the famous Singapore Sling cocktail at the Long Bar.
\end{itemize}
\vspace{-0.25cm}
\textbf{Hawker Centers:}
\vspace{-0.3cm}
\begin{itemize}
    \item \textbf{Maxwell:} A popular hawker center (and one of the largest) with lots of great food. Here, you could find the famous Tian Tian Hainanese Chicken Rice.
    \item \textbf{Lau Pa Sat:} A historic hawker center in the central business district.
    % \item These two are probably the most popular (and touristic) hawker centers, but there are essentially an infinite number to explore in total.
\end{itemize}
\vspace{-0.25cm}
\textbf{Nightlife:}
\vspace{-0.3cm}
\begin{itemize}
    \item \textbf{Clarke and Boat Quay:} A popular nightlife area with lots of bars. This was basically the ``Andra-Lång of Singapore'', with cheap beer and great vibes. The bar Georgetown was a popular place among exchange students.
    \item \textbf{Cè La Vie:} A popular nightclub with a great view of the city from the top of Marina Bay Sands (don't drink too many beers here, or you'll be financially ruined).
    \item \textbf{Zouk and Marquee:} Nightclubs.
\end{itemize}
\vspace{-0.15cm}
Lastly, if you're in Singapore during autumn, don't miss \href{https://singaporegp.sg/en/}{the Singapore Grand Prix} (a Formula 1 night race). Unfortunately, as I was there during spring, I didn't get to experience this. For the general sports nerds, the \href{https://www.livgolf.com}{LIV Tour}, as well as \href{https://www.svns.com/en}{a large Rugby tournament}, was hosted in the city during my time there.
\vspace{-0.15cm}
\begin{figure}[H]
    \centering
    \includegraphics[width=0.99\linewidth, trim={0 0.3cm 0 0}, clip]{figs/sg.pdf}
    \vspace{-0.1cm}
    \caption{\raggedright Despite its small size, Singapore had some must-do activities and places to visit. Clockwise from the top left: Juwel Changi Airport, Marina Bay Sands, Sentosa Island, the view from Cè La Vie nightclub, Gardens by the Bay, and Buddha Tooth Relic Temple.}
\end{figure}

As for the cultural aspects, Singapore was a multicultural country with lots of different cultures and religions. The official language is English, but you'll also hear lots of Mandarin and other languages. For example, the locals spoke a very special version of English called Singlish, which was a mix of English and some other local languages. This could be a bit hard to understand at first, but you got used to it. The country also felt extremely safe, and I never felt uncomfortable. This could for example be seen at school, where everyone left their laptops and bags unattended in the libraries or canteens, which is something that'd rarely happen in Sweden. 
\vspace{-0.15cm}
\section*{Asia -- a Student-Friendly Continent}
\phantomsection
\label{asia}
\vspace{-0.1cm}
\addcontentsline{toc}{section}{Asia -- a Student-Friendly Continent}
In many ways, Asia was the perfect continent for students. The cost of living was usually low, the food was amazing and there was lots to see and explore. When your base was Singapore, you could usually find lots of options for cheap flights to many neighboring countries. I traveled to Malaysia, Thailand, Vietnam, Indonesia, and Hong Kong during the school semester, and then Laos, Japan, South Korea, and Taiwan before I went back to Sweden. Below, I've shortly summarized my trips and added some tips for each destination. Note that cash still was king in most Asian countries, so I'd strongly recommend to always carry some cash with you, especially in the Southeast Asian countries.

\textbf{Malaysia:} I visited Kuala Lumpur for a short time, but I really liked the city, which was both modern and packed lots of culture. Unfortunately, I missed the Batu Caves, which I've heard is a must-see, so try to make time for that when in KL. I also spent one day in Johor Bahru (JB), a city just across the border from Singapore, and a great place for a day trip (this was actually many Singaporeans favorite weekend destination for shopping and eating out). Malaysia was also very cheap in general, and the food was amazing.

\begin{wrapfigure}{r}{0.25\textwidth}
    \centering
    \vspace{-0.55cm}
    \includegraphics[width=0.24\textwidth]{figs/thaibeach.pdf}
    \vspace{-0.1cm}
    \caption{\raggedright Beach on Koh Lanta.}
    \vspace{-0.32cm}
\end{wrapfigure}
\textbf{Thailand:} I went to Koh Lanta and Koh Phi Phi in Krabi for a week of ``studying'' and relaxing. Thailand was one of my favorite countries -- it was cheap, had amazing food and people, and the weather and beaches were great. Apart from Koh Lanta and Koh Phi Phi, there were lots of other alternatives to choose from, such as Phuket, Koh Samui, and Koh Phangan. Nevertheless, I'd recommend to visit Koh Lanta, which felt chill and relaxed. Koh Phi Phi was a bit more touristy, but had some amazing beaches and views. I've also heard that Bangkok is great, and also Chiang Mai in the north. Unfortunately, I didn't have time for these.

\begin{wrapfigure}{r}{0.25\textwidth}
    \centering
    \vspace{-0.55cm}
    \includegraphics[width=0.24\textwidth]{figs/ninhbinh.pdf}
    \vspace{-0.12cm}
    \caption{\raggedright The landscape in Ninh Bình.}
    \vspace{-0.32cm}
\end{wrapfigure}
\textbf{Vietnam:} I was in Vietnam for a week during recess, and visited Hanoi, Ninh Bình, and H\d{ô}i An. Vietnam was also one of my favorites, with friendly people, amazing food, and lots of culture. Hanoi was very busy and lively with culture, history, and lots of great food. On the contrary, Ninh Bình was more nature-focused. There, we went on a canoe trip through the many caves and mountains in the area. H\d{ô}i An was a small historical town known for its tailor stores, perfect for getting custom-made clothes. If I had time, I also would've liked to visit Ho Chi Minh City in the south, known for its food and history from the Vietnam War.

\begin{wrapfigure}{r}{0.37\textwidth}
    \centering
    \vspace{-0.15cm}
    \includegraphics[width=0.36\textwidth]{figs/hk.pdf}
    \vspace{-0.18cm}
    \caption{\raggedright The Hong Kong Island skyline.}
    \vspace{-0.32cm}
\end{wrapfigure}
\textbf{Hong Kong:} I also went to Hong Kong for a weekend and really liked it. In many ways, the city felt similar to Singapore, but I believe that HK had more to offer and the city felt more lively. For instance, I visited Victoria Peak, which had a great view of the city. We also hiked along the Dragon's Back trail, which is a must-do when in HK (unfortunately, we didn't have the weather gods on our side). Other than that, I'd recommend exploring the city and finding your own favorite spots. We had a very interesting experience at Mr. Wong's, a restaurant in the city. For 100 Hong Kong dollars, you got to eat (and drink) as much as you wished, and the place was popular among exchange students.

\begin{wrapfigure}{r}{0.29\textwidth}
    \centering
    \vspace{-0.53cm}
    \includegraphics[width=0.28\textwidth]{figs/nusapenida.pdf}
    \vspace{-0.1cm}
    \caption{\raggedright Kelingking Beach.}
    \vspace{-0.32cm}
\end{wrapfigure}
\textbf{Indonesia:} I went to Batam during a weekend, which is a small island only 45 minutes away from Singapore by ferry. As the rest of Indonesia, Batam was very cheap, and we had a great time just relaxing and studying in a large villa. I also visited Bali for a week with my family. Here, we first explored Ubud, known for its nature (try the river rafting!), many waterfalls, rice terraces, and temples. Then, we went to Nusa Penida, a small island with amazing views and beaches (Kelingking is the Instagram king of beaches), and later Seminyak, a beach town in the south. Here, we tried surfing which was lots of fun (but hard). We also visited the Uluwatu Temple (beware of the monkeys!), where we witnessed a weird local play. If you want a mix of culture, relaxing, and adventure, Bali is probably the perfect place.

\begin{wrapfigure}{r}{0.27\textwidth}
    \centering
    \vspace{-0.53cm}
    \includegraphics[width=0.26\textwidth]{figs/laos.pdf}
    \vspace{-0.1cm}
    \caption{\raggedright Hot air balloon tour in Vang Vieng.}
    \vspace{-0.35cm}
\end{wrapfigure}
\textbf{Laos:} My first stop after school ended was Laos. I believe that Laos is one of the most underrated countries in Asia. It was very cheap, the people were very friendly, and the nature was incredible. We flew to the capital Vientiane and spent half a day there (this city didn't offer too much, so I probably wouldn't spend more than one or two days here). From here, we took a mini bus to Vang Vieng, a small town known for its nature and adventurous activities, where we spent three nights. Here, we went on a hot air balloon tour (this was a must-do!). We also rented a motorcycle for a cheap price, which without a doubt was the best way to explore the area. The local taxi service \href{https://www.google.com/url?sa=t&source=web&rct=j&opi=89978449&url=https://apps.apple.com/us/app/loca-laos-e-mobility-payment/id1374380531&ved=2ahUKEwiT9L-zvJaOAxVMKBAIHf7-CVoQFnoECBcQAQ&usg=AOvVaw16TeN9fX6dDoffHbK5roM6}{Loca} was an alternative, but this gave you less freedom. We explored caves (where you could try water tubing), blue lagoons (the best places to chill and cool off in the heat, there were several of these in the area), and hiking spots. Lastly, we took the train to Luang Prabang, a UNESCO World Heritage Site known for its temples and culture, where we spent two nights. Here, we also rented a motorcycle and visited the Kuang Si Waterfalls, which is a must-see when in Laos. The waterfalls are beautiful, and you could also swim in the turquoise water. I also recommend the night market and the bowling alley, where you could play bowling and drink beer for a cheap price. As a side note, the railway was actually very modern and efficient in Laos, mostly due to Chinese investments in the country.

\begin{wrapfigure}{r}{0.28\textwidth}
    \centering
    \vspace{-0.15cm}
    \includegraphics[width=0.27\textwidth]{figs/tokyo.pdf}
    \vspace{-0.15cm}
    \caption{\raggedright The view from Tokyo Skytree (it was the same in every direction).}
    \vspace{-0.3cm}
\end{wrapfigure}
\textbf{Japan:} After Laos, I spent 10 days in Japan, where we visited Tokyo, Mt. Fuji, Kyoto, Nara, and Osaka. Japan is a unique country, with a mix of modern and traditional culture. The public transport system was creepy good, and you could easily get around using the Shinkansen (bullet train) or subway. For the latter, I'd recommend the Suica card (you should be able to access this through Apple Wallet, just tap ``add card -- travel card'' in the app) which could be topped up via Apple Pay. We started with four nights in Tokyo, which, except for being the largest city in the world, was very busy and lively with lots of culture and amazing food. Some of the highlights include the Senso-ji Temple, Shibuya Crossing (the world's busiest pedestrian crossing), Imperial Palace, Meiji Shrine, Tokyo Skytree (for a view of the endless concrete), Takeshita Street, and the different parks. You should also check out Golden Gai, a small area with lots of bars and restaurants, and the \href{https://www.teamlab.art/e/tokyo/}{teamLab Planets museum}.

\begin{wrapfigure}{L}{0.26\textwidth}
    \centering
    \vspace{-0.57cm}
    \includegraphics[width=0.25\textwidth]{figs/japan.pdf}
    \vspace{-0.25cm}
    \caption{\raggedright Sunset over Kyoto from Fushimi Inari Shrine and incredible ramen in Tokyo.}%and my \\ favorite ramen.}
    \vspace{-0.7cm}
\end{wrapfigure} 

After Tokyo, we took the local train to Fuji, where we spent one night. Unfortunately, we didn't have the best weather, but we still got to see the mountain, which is a must-see when in Japan. Anyway, the area was very beautiful, and it was nice to have a date with nature after a few days in the largest city in the world. Next, we took the Shinkansen to Kyoto, where we spent three nights. Kyoto is known for its temples and culture, and some of the highlights include the Fushimi Inari Shrine (the famous red gates), Bamboo forests, and Gion district. We also visited Nara, a small city just outside of Kyoto, which is known for its parks with free-roaming deer. Lastly, we took the local train to Osaka, where we spent one night. Osaka is mostly known for its food and nightlife. I definitely recommend the Space Station bar, where you could play old video games while having a beer.

In general, I'd recommend exploring the cities, finding your own favorite spots and experiencing the culture. Regarding food, you should probably try everything (since it was all great), such as okonomiyaki (a savory pancake), takoyaki (octopus balls), sushi, nigiri, ramen, and wagyu beef (we tried an A5-ranked wagyu beef in Osaka, which was expensive, but probably one of my best food experiences ever).

\begin{figure}[H]
    \centering
    \includegraphics[width=0.99\textwidth]{figs/japan2.pdf}
    \caption{\raggedright From the left: Shinjuku Gyoen National Garden in Tokyo, Nara Park, okonomiyaki in Kyoto, and A5-ranked wagyu beef in Osaka.}
\end{figure}
\vspace{-0.6cm}

\begin{wrapfigure}{r}{0.23\textwidth}
    \centering
    \vspace{-0.08cm}
    \includegraphics[width=0.22\textwidth]{figs/seoul.pdf}
    \vspace{-0.1cm}
    \caption{\raggedright Gyengbokgung Palace in Seoul.}
    \vspace{-0.35cm}
\end{wrapfigure}
\textbf{South Korea:} After Japan, I flew from Osaka to Seoul, where I spent four nights. Honestly, South Korea felt like Japan's little brother (as we Swedes know, little brother $<$ older brother when it comes to countries -- but to be fair, most things would feel like a downgrade after Japan), however it was still well worth a visit. Here, some of the highlights include the Gyengbokgung Palace, Bukchon and Ikseon Hanok Villages, Seoul Tower (try to be there at sunset), Cheonggyecheon Stream (especially at night when they turned on the lights), and the different markets (especially Myengdong and Gwangjang). I also visited the DMZ, a strip of land that separates North and South Korea. This was a very unique experience, and I'd definitely recommend going there if you have the chance. I booked a tour via \href{https://www.getyourguide.se/korean-demilitarized-zone-l89058/?visitor-id=CC8689D2A4F84D8A809D8C526DAF718E&locale_autoredirect_optout=true}{GetYourGuide}, which worked well. Regarding transportation, the subway system in Seoul was very good and efficient. The best option for this was to get a T-money card at any convenience store, which could be topped up using cash. Moreover, Google Maps worked poorly in South Korea, and most locals used \href{https://www.google.com/url?sa=t&source=web&rct=j&opi=89978449&url=https://apps.apple.com/us/app/kakaomap-korea-no-1-map/id304608425&ved=2ahUKEwj33J_Jm86NAxV5ExAIHUB2NpoQFnoECCMQAQ&usg=AOvVaw2_J4lhArArkKe5YWiU-7Jk}{KakaoMap} instead. Lastly, the food here was great, and I'd recommend to try all the local dishes, like Korean BBQ, bibimbap, gimbap (Korean sushi), buchimgae (Korean pancakes), and tteokbokki.

\begin{wrapfigure}{r}{0.27\textwidth}
    \centering
    \vspace{-0.53cm}
    \includegraphics[width=0.26\textwidth]{figs/taiwan.pdf}
    \vspace{-0.1cm}
    \caption{\raggedright The view from Elephant Mountain in Taipei.}
    \vspace{-0.35cm}
\end{wrapfigure}
\textbf{Taiwan:} My last stop before going back to Sweden was Taiwan, where I spent five nights in Taipei. In my opinion, Taiwan felt like a less developed version of Hong Kong, with lots of Chinese-influenced culture (I suspect that is as close as you get to China without actually going there). Here, you should definitely include the Taipei 101 skyscraper (try to get there at sunset), night markets (Ninhxia and Raohe night markets were great), Chiang Kai-shek Memorial Hall, Longshan Temple, Elephant Mountain (for a great view of the city), and the Maokong district (known for its tea culture and nature) in your itinerary. I also spent a day in Jiufen, a small town in the mountains with lots of history (the old street is probably the most famous attraction here), culture (try the local tea!), and shopping. Also, you must try the different food options -- dumplings, braised pork, stinky tofu (if you dare, it didn't taste that bad in my opinion), and bubble tea. Lastly, the public transport system in Taipei was great, and you got around easily using the subway or buses. I'd recommend getting an EasyCard at one of the subway stations, which worked similarly to the T-money card in Seoul.

\vspace{-0.25cm}
\hrulefill

Apart from the countries above, there's of course a lot of other places to visit in Asia. From what I've heard, the Philippines is a great place, as is Angkor Wat in Cambodia, and Sri Lanka. China is also definitely worth a visit, however as of now, Swedes have to apply for a VISA and I believe that this process isn't the smoothest. Hopefully, this will change for future students. Lastly, Australia and New Zealand are definitely worth a visit, and not too far away from Singapore. However, then you'll have to spend more time (and money), which I personally didn't have. As for the countries I visited, I really liked all of them, and I think that they all had something unique to offer (i.e. visit them all if you can!). However, if I had to choose three favorites (extremely hard), I'd say Japan, Vietnam, and Laos, with all others in a close fourth place. These countries were very different from each other, but they all had amazing food, culture, and nature.
