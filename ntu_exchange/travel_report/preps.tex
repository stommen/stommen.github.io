\chapter*{Preparations}
\addcontentsline{toc}{chapter}{Preparations}
\vspace{-0.6cm}
As you might understand, there was some preparation work you had to complete before traveling across the globe. However, just like my experience with Swedish and Chalmers culture, I found that Singapore and NTU were very well organized (except for the course selection process, which I'll address later). All the necessary steps in the application process were clearly outlined. I have friends who went to other countries where equivalent processes seemed less straightforward, from my understanding. Either way, I hope that the following sections make the application process smoother and provide some additional tips and tricks. I tried to write this as I completed each step to ensure that the information was fresh in my mind. That said, keep in mind that things can change from year to year, so this shouldn't be treated as the ``definitive truth''. Always follow the official instructions from NTU, and if you notice any discrepancies between my explanations and NTU's, assume that their version is correct. Lastly, note that the deadlines I mention below may differ depending on whether your exchange starts in autumn or spring.
\vspace{-0.25cm}
\section*{The Application Process}
\phantomsection
\label{app}
\addcontentsline{toc}{section}{The Application Process}
I was nominated by Chalmers in early September and received an email from NTU shortly thereafter. As an Engineering Physics student, it was somewhat unclear to the Chalmers coordinator whether to nominate me to the College of Engineering (CoE) or the College of Science (CoS). Ultimately, another Physics student and I decided that the CoS suited us best. It wasn't entirely clear what being nominated to a specific college actually meant. From what I understood, the main requirement was to ensure that at least 50 \% of your course credits came from the college you were nominated to. However, I'm not sure if this was checked in the end (probably not, as I believe that I failed to achieve that). Nevertheless, as a Physics student, I was happy with my decision (the other Chalmers students were nominated to the CoE), as the most relevant courses for me were offered at the CoS. For most MSc programs at Chalmers, I'd assume that the CoE is the ideal choice, and it would likely work for physics-related MSc programs as well. 

Additionally, whether you were nominated as a graduate or an undergraduate student also came with some catches. If given the choice, I'd recommend asking to be nominated as an undergraduate (which I believe is the default), as this allowed you to apply for both post- and undergraduate courses. This wasn't the case if you were nominated as a postgraduate student. As a postgraduate nominee, you'd likely face issues when applying for undergraduate courses. Moreover, during my exchange term, there weren't many postgraduate courses available to exchange students at all, which further supports being nominated as an undergraduate.

Going back to the application -- in the said email received after the nomination, NTU provided a link to the application portal together with accompanying instructions. Apart from some weird queries, e.g., entering your ``race'', the application form mostly asked for basic information: contact information, medical status, special needs etc.. You also needed to provide a photo of the bio-data page in your passport (i.e. the ``hard'' page with your photo, \textit{make sure to save this for later!}) and a photo of yourself cropped to a specific size (\textit{also save this!}). For the latter, I used a ``passport-photo-maker'' app (link \href{https://www.google.com/url?sa=t&source=web&rct=j&opi=89978449&url=https://apps.apple.com/se/app/passport-size-photo-maker-app/id1615533705&ved=2ahUKEwjq7fmg14eKAxXlIRAIHYyYDKcQFnoECEIQAQ&usg=AOvVaw2MFZUfJ5qukJls3hL6DmwZ}{here}), which seemed to work well. You also had to provide your English certificate (I received mine a couple of months earlier via my Chalmers email) and your Transcripts of Records (ToR). Lastly, you were asked to enter 10 courses that you found interesting. You might hear otherwise, but from my understanding, these only served as a ``check'' that you'd done some research on the NTU coursework -- you could easily swap freely among more courses later on (see \hyperref[courses]{Course Selection and Registration}).

Regarding the ToR, Chalmers Servicecenter started signing these electronically some time before our application deadline. Although, NTU demanded these to have the Chalmers seal, which the electronic sign didn't provide. Make sure to check if this is still the case for you! I know some students who didn't want to wait for the coordinators to fix this, so they printed their ToR and went to Chalmers Servicecenter to sign them. So, this is definitely an option if this applies to you and the coordinators are slow for some reason. Overall, the instructions made the application form rather easy and straightforward to complete, and if these were followed, your application was most likely approved.
\section*{After Being Accepted}
\phantomsection
\addcontentsline{toc}{section}{After Being Accepted}
If you'd done everything correctly in your application, access to the Exchange/Study Abroad Portal (SAP) was granted in the beginning of November. This portal contained essentially everything you needed to know and prepare prior to your exchange. There was also a pre-arrival meeting held on Teams in the second half of November. I'd argue that, as I claimed, the portal itself provided all the necessary information, however the meeting was a nice way of hearing everything at once. Also, a link to a Telegram group was provided, where all incoming exchange students to NTU could ask each other questions about the application process. However, this later evolved into a way of getting in touch with potential travel partners or maybe find some people up for a beer.

Inside the portal, the first things you had to do before a deadline were to accept the NTU exchange offer, download the Letter of Enrolment (print this, since it might be needed when entering Singapore) and apply for campus housing (if you wanted). Regarding housing, you can read more about my experiences in the \hyperref[house]{Housing} section in the next chapter. After accepting the offer, you could also activate your NTU email and network. From that point onwards, NTU recommended using that email for all future communication to avoid ending up in spam. Next, you had to apply and appeal for additional courses beyond the 10 selected in your initial application and also submit your Student's Pass (STP) application. You can read more about this in the two following subsections. Moreover, something I couldn't find explicitly stated in the SAP (at least not clearly), was that you also had to submit your Singapore Arrival Card (SGAC) 3 days prior to your arrival, which was easily done on \href{https://eservices.ica.gov.sg/sgarrivalcard/}{ICA's website}.

Some things also had to be completed after arriving in Singapore. This included completing the STP application (see below), collecting your matriculation card (essentially like the Chalmers Kårkort) and paying the NTU registration and miscellaneous fees. However, as in the general case, these steps were well instructed and easily done.

\subsection*{Student's Pass Application/VISA}
\phantomsection
\addcontentsline{toc}{subsection}{Student's Pass Application/VISA}
From mid-November until mid-December, you had to submit your Student's Pass (STP) application. This served as proof of your legal stay (similar to a VISA) and it was mandatory for all international full-time students. The instructions on how to do this were clearly described in the SAP. I've summarized the steps below for an overview:
\begin{enumerate}
    \item First, you had to download the ``SOLAR form'' from the SAP, which contained your personal credentials to access the STP application interface on \href{https://eservices.ica.gov.sg/solar/index.xhtml}{ICA's} website. 
    \item Next, you had to log on to the STP application page and complete the ``eForm16 form''. Many queries in this reminded much of those in the initial application. I also know that one query demanded your signature. However, I missed this and my application was approved anyway. But to be safe, I'd probably fill this in as well.
    \item Then, you had to upload the eForm16 together with a photo of yourself and the bio-data page of your passport, hence why you should save these during your initial application (don't delete them after this either though, as they'll probably come in handy some other time during your exchange, eg., while traveling). However, the photo dimension requirements differed (at least for me) from those in the initial application. Thus, I used an online \href{https://imageresizer.com}{photo cropper} to resize my photos. 
    \item Before your application was processed, you had to pay a processing fee. For me, this was 45 Singapore dollars (SGD). Then, if and when your application was approved (which took a few days), you had to log on again and print your so-called IPA-approved STP, which you might have to show when entering Singapore.
    \item Lastly, you were able to complete two of the final three steps. First, you had to approve, download and then upload the form of conditions and terms regarding the STP. When this had been processed (this also took a couple of days), you had to pay the second processing fee (which for me was 90 SGD). The third and last step was completed after arriving in Singapore, see below.
\end{enumerate}
The IPA-approved STP could be used to enter Singapore \textit{once}, and was then finalized after arriving at NTU, by booking a meeting with ICA when they visited campus during the first week. For this occasion, you needed the E-pass you receive when entering Singapore, a physical passport photograph (for which I used a photo booth in Sweden), your passport, proof of your payments, your IPA and a Singaporean phone number. How to get the latter is covered below in the \hyperref[sims]{SIM Card} section. After completion, you could travel in and out of the country as you wished and also create your Singpass. This was needed for creating a local bank account, as described in the \hyperref[eco]{Economy} section (however, it was convenient having this even if you didn't apply for a bank account, since it essentially worked as a digital ID in Singapore, much like BankID in Sweden).

\subsection*{Course Selection and Registration}
\phantomsection
\label{courses}
\addcontentsline{toc}{subsection}{Course Selection and Registration}
This was probably the part that gave me the most headache during the whole application process. I'd say that our Swedish ``Antagning'' system suddenly seemed very simple and effective after going through what appeared to be the Singapore (or at least NTU) equivalent. However, I should add that if you did everything calmly and as instructed, there most likely weren't any issues. To make everything as clear as possible, I've listed everything that had to be done below. If I'd gotten a list like this when I was going through this process, I believe that some (if not all) questions I had then would've been crystal clear, thus I think that this is an appropriate way to describe this. 
\begin{enumerate}
    \item[i)] First, you had to choose 10 courses during your initial application (as mentioned before). These are just a first centerpiece of courses you'd like to attend, and you could easily add more later on.
    \item[ii)] Then, when you got access to the SAP (or closely afterwards), you knew 
    \begin{enumerate}
        \item[a)] which of the 10 courses you got approved for, and
        \item[b)] which of these that were actually offered.
    \end{enumerate}
    Don't worry if you get many rejections at this time -- that's what the following steps are for (as for me, I was only approved for one course at this stage).
    \item[iii)] Next, the course E-request opened for the first time. This was your first time to shine, i.e. you could now \textit{request} for additional courses. If a course had pre-requisites, you could add links and contents of courses you'd read at Chalmers, ensuring that you met these. At this time, the definite course schedule had been uploaded, so you could be sure which courses were actually available.
    \item[iv)] By the same time as iii), you could also \textit{appeal} for rejected courses until the end of the notorious Add/Drop Period, see vii). This meant that you could be approved for courses marked as ``rejected'' in the SAP, if you sent an email(s) to the college(s) offering the course(s) as instructed. This will most likely be the case for some of your 10 initially chosen courses, as discussed above.
    \item[v)] Then, after the first E-request period had closed, you were asked to rank your (at that time) approved courses. This allowed you to be allocated (registered) for up to 5 courses prior to the semester start, which could decrease the amount of work during the Add/Drop Period.
    \item[vi)] The E-request period then opened a second time closer to the start of the term, at which you could send in requests for even more courses as described in iii). 
    \item[vii)] Lastly, at the same time as vi), the two-week-long Add/Drop Period started. You might've heard horror stories about this, but if you did everything as instructed, you were most likely fine. Here, you could add and drop approved courses freely to your schedule through STARS (link below), and appeal for rejected courses as described in iv). If you were lucky, you were happy with your schedule after the first E-request period. Then, you could sit back and relax during this time (after you'd added the courses to your schedule, of course). However, if you ended up like me, you had to spend some time during this period to get the schedule you wanted.
\end{enumerate}
I know that the different terms used here can be very confusing. I think that the most usual (and important) misunderstanding was the difference between an \textit{``approved''} and a \textit{``registered''} course. An \textit{approved} course was one that you'd been accepted to, while a registered course was one that you'd added to your schedule in STARS and for which you'd been allocated a spot. The latter was the holy grail of attending a course -- and you achieved it by adding an approved course to your schedule during the Add/Drop Period.

Finally, I'd recommend maintaining close communication with the Director of Studies (DoS) at your MSc program, in order to make sure that you actually can accredit the courses you choose. Don't forget to send your study plan (containing the courses you plan to attend) to your DoS. Since we didn't receive any information about this, I must thank my friends for reminding me on this (you know who you are). The study plan could be downloaded under ``Forms and manuals'' \href{https://www.chalmers.se/en/education/your-studies/exchange-studies-and-international-opportunities/exchange-studies/preparing-for-exchange-studies/}{here}. I sent the first version of this when I was nominated back in September. At this time, the definite course list and -schedule usually haven't been released for the upcoming term(s). However, you could use the course content website (link below) to check which courses were given the corresponding term(s) previous years as an indication of which courses will be available to you. As the course selection process starts and progresses, you can use the study plan to update your DoS about this, assuring that the courses you choose are valid for your program.

\vspace{-0.3cm}
\hrulefill

Below, I've listed two useful websites in order to survive the somewhat frightening journey of the course selection process:
\begin{itemize}
    \item \href{https://gem.ntu.edu.sg/index.cfm?FuseAction=Programs.ViewProgramAngular&id=10006}{General coursework information for exchange students.} This site contained all the necessary links and information when it came to choosing courses -- from the course catalogue (containing courses from previous years) to lists of restricted courses.
    \item \href{https://wish.wis.ntu.edu.sg/pls/webexe/ldap_login.login?w_url=https://wish.wis.ntu.edu.sg/pls/webexe/aus_stars_planner.main}{STARS}. Note that this was only available after you'd activated your NTU account. Here, you could build and customize your own schedule for the upcoming term(s). This was also where you built your schedule during the Add/Drop Period.
\end{itemize}
\vspace{-0.35cm}
\section*{Vaccinations}
\phantomsection
\label{vacc}
\addcontentsline{toc}{section}{Vaccinations}
When staying for longer periods in Singapore, there were definitely some recommended vaccines worth considering (especially if you intended to visit other parts of Asia). I received Twinrix (Hepatitis A \& B), Typhoid, Dukoral (against Cholera), and Japanese encephalitis. However, I strongly recommend that you visit a health- or vaccine-center to get a tailored recommendation based on your personal travel plans and medical needs. I visited \href{https://vaccindirekt.se/mottagningar/heden-gbg/}{VaccinDirekt} at Heden and they were very friendly and professional. They also had student discounts on almost all vaccines. Note that some vaccines require additional doses for full protection, so I wouldn't procrastinate on this. I received my first doses in mid-November, i.e. about one and a half months before I left, which was -- expressed in pure Swedish -- ``lagom framförhållning'' (literally ``sufficiently good timing''). I also recommend buying malaria pills if you plan to travel to areas affected by malaria. This was also provided by VaccinDirekt and I bought 12 pills (enough for one trip to a malaria-affected area). From what I'd heard, the pills were cheaper in Sweden, and you also avoided potential scams by not buying them somewhere in Asia.   
\section*{Insurance}
\phantomsection
\addcontentsline{toc}{section}{Insurance}
As a Chalmers student, you were insured through Kammarkollegiet (\href{https://www.kammarkollegiet.se/vara-tjanster/forsakring-och-riskhantering/forsakringar-for-studier-och-utlandska-besokare/utresande-utbytesstudenter-student-ut}{Student UT}). This insurance covered essentially everything while you were in Singapore (including travel to and from the country). However, if you planned to travel beyond Singapore, additional travel insurance was recommended. Most home insurance policies, if you had one, included 45 or 60 days of travel coverage, so you might've needed to purchase extra insurance to cover the remaining period. Then, you were basically faced with two options:
\begin{itemize}
    \item[a)] buy travel insurance before every trip separately, or
    \item[b)] buy one travel insurance to cover the whole period from the end of your home insurance's coverage. 
\end{itemize}
I chose option b), using ERV's \href{https://www.erv.se/privat/vara-reseforsakringar/reseforsakring-ung/}{``Reseförsäkring Ung''}, which cost me a few thousand SEK. I was happy with this decision, as it likely saved me the trouble of buying insurance every other weekend. From my calculations, the price would've been around the same for option a), though I can't assure that my calculations were correct :). This, of course, also depends on how much you intended to travel, but if you planned to travel a lot, chances were that you'd save money if you bought only one insurance (and as said, you also saved some extra work before each trip). I know from previous travel reports that many have followed option a) though, so this probably would've worked as well. Either way, you should always check the conditions and what's covered before buying insurance.
\section*{SIM Card}
\phantomsection
\label{sims}
\addcontentsline{toc}{section}{SIM Card}
Some other students from Chalmers and I chose to buy a SIM card directly at the airport. Since you hadn't received your STP at this point, you were only allowed to buy a 30 day prepaid SIM card. After you'd fixed your STP, you could buy a postpaid plan. However, buying the SIM at the airport definitely wasn't the cheapest alternative, as we payed 50 SGD for 100 GB of data (some roaming included) and one of our peers got the same deal for 14 SGD at campus. So, if you were able to make it to campus without internet (which you should be able to, either by Grab or MRT), this was definitely the most economic option. As for the choice of network operator, I initially used M1 (most people I've talked to did as well), that had OK connection, but not more. When I received my Singpass, I switched to an eSIM from Simba that offered a great deal with 100 GB of data, including 7 GB of roaming in other Asian countries (this was later increased to 500 GB of data and 18 GB of roaming) for 12 SGD per month. However, the connection was on the same level as with M1, i.e. not great. I know that some students used Singtel, and experienced a better connection (the roaming and total data probably wasn't as generous as with Simba though). As a final note, if your SIM doesn't include roaming in a specific country while traveling, I'd recommend using \href{https://www.google.com/url?sa=t&source=web&rct=j&opi=89978449&url=https://apps.apple.com/ca/app/roamify-travel-esim-data-call/id6472396976&ved=2ahUKEwi7y9-96omOAxUIJhAIHZvSAG0QFnoECAsQAQ&usg=AOvVaw2EyfUaHaPZfjNMj43MYdyp}{Roamify}. This allowed you to buy eSIM data packages for most countries, and worked great for me in Japan, South Korea, and Taiwan, where my Singaporean SIM card plan through Simba didn't include much roaming.
\section*{Traveling to and from Singapore}
\phantomsection
\addcontentsline{toc}{section}{Traveling to and from Singapore}
I bought my flight tickets with Lufthansa in mid-October together with two other students from Chalmers and traveled on the 6th of January. By October, there were only a few tickets left on that particular flight (at least according to Lufthansa). However, as of writing this (on December 1st), tickets are still available -- though at a higher price. That said, I strongly recommend buying your tickets as early as possible, as this will likely secure a better price and give you peace of mind knowing at least one problem is solved. While NTU advised us to wait until completing our STP applications before booking the flights, I don't think that many students actually followed this recommendation.

Nevertheless, we opted for a round-trip ticket with a flexible return (i.e. reschedulable) to allow more freedom for any potential travel plans before heading home. This actually came in handy for me, as I decided to continue exploring Asia for another 10 days after my initial return date. From an economic perspective, this seemed roughly equivalent to buying a return ticket closer to the return date. In terms of price, Lufthansa offered the cheapest tickets at around 8\,000 SEK (for the entire round trip), but I know that some Chalmers students flew with Turkish Airlines at a (assumingly) similar cost. Many airlines, including Lufthansa and TA, also provided student discounts and benefits, so be sure to check for those as well. From the airport to campus, there were essentially two options: Grab or MRT/bus. We used the MRT, which worked fine and was the cheaper (but slower) option. However, you had to change to the campus bus (at Pioneer station) or 199 bus (at Boon Lay station) in order to get to campus. Still, Grab was most likely the best option (especially if you could share one), since you had lots of luggage and were new to town (Grab was also available 24/7).

\vspace{-0.3cm}
\hrulefill

According to NTU, you needed the following when entering Singapore (in printed form):
\begin{itemize}
    \item (Obviously) Your passport
    \item The Letter of Enrolment
    \item The IPA letter from your STP application
\end{itemize}
However, I only had to show my passport as usual, and also submit the Singapore Arrival Card (SGAC) as explained in \hyperref[app]{The Application Process} section.
\vspace{-0.1cm}
\section*{Scholarships}
\phantomsection
\addcontentsline{toc}{section}{Scholarships}
This part actually emerged as a big (positive) surprise to me. When reading previous travel reports, I stumbled upon multiple tips of scholarships that one as an upcoming exchange student could apply for. However, I was almost 100 \% certain that I wouldn't receive any of them. This was -- drum-roll -- not the case! I received several of the scholarships I applied for. That said, I don't want to make it sound too easy -- I did put effort into writing personal letters and gathering the necessary documents for each application, though I also used some tricks to make the process more time-efficient.

As I traveled during spring, most of my applications were due during autumn. That made summer the perfect time for me to complete the skeleton of my scholarship applications. I prepared the following documents:
\begin{enumerate}
    \item[--] A personal letter -- introducing myself, explaining the reason behind my application, and why I'd be a good candidate.
    \item[--] An updated version of my CV.
    \item[--] A preliminary budget -- including expenses and incomes during the exchange period. Here, you must do what we physicists call \textit{an order-of-magnitude estimate}. I mostly studied previous travel reports, and for the everyday living expenses in Singapore, NTU had a nice summary \href{https://www.ntu.edu.sg/eee/admissions/programmes/graduate-programmes/international-students}{here}. However, as I provide my budget in the end of this report, you can also use that as a reference.
\end{enumerate}
\vspace{-0.3cm}
I must praise myself here -- preparing these turned out to be extremely smart, and it definitely saved me lots of time when the applications opened. This was crucial, since I had several other things on my mind during this time -- from quantum mechanics and computational physics to the whole NTU application process.

Although I'd raise a finger of warning here, as the scholarship organizations are very strict with their application due dates, and these might vary depending on if you travel during autumn or spring. So, I'd definitely recommend to check the different options as quickly as possible to put the due dates in your calendar and adjust your application-writing process from there.

\hrulefill

In total, I applied for six scholarships:
\begin{itemize}
    \item \href{https://www.globalgrant.com/sos-stipendier}{\texttt{SOS-stipendiet}}
    \item \href{https://www.whitlocks.se/anna-whitlock/}{\texttt{Anna Whitlocks Minnesfond}} (both for ``Masterstudier i utlandet'' and ``Postgymnasialt i utlandet'')
    \item \href{https://stiftelseansokan.seb.se/sbs/carleriklevin/grandid/user/login/selectidp/bankid.deviceChoice}{\texttt{Carl Erik Levins Stiftelse}} (try \href{https://stiftelseansokan.se}{this link} if the first one doesn't work)
    \item \href{https://www.felixneubergh.se}{\texttt{Doktor Felix Neuberghs Stiftelse}}
    \item \href{http://stipendieguiden.com/listing/stiftelsen-aaa/}{\texttt{Stiftelsen AAA}} (send an email to receive information regarding the application procedure)
    \item \href{https://www.sverigesingenjorer.se/medlemskap/stipendier/}{\texttt{Sveriges Ingenjörer}} (``Understöds- och stipendiefonden för utlandsstudier'', also note that you must've been a member for at least 6 months at the time of your application, so if you aren't a member, apply for a membership ASAP)
\end{itemize}
I also knew that \href{https://www.asemduo.org/02_programs/programs_02.php}{\texttt{ASEM-DUO}} had a partnership between Singapore and Sweden that provided financial help for exchange students. However, I missed the deadline on this one so I can't say much about the application process. In \hyperref[appB]{Appendix B}, I've listed some more potential scholarships that you can apply for. Lastly, note that if you receive a scholarship, you might have to report back to the organization after your exchange. This is usually done by writing a short report about how you've used the scholarship. In some cases, you might also have to show receipts, so I'd recommend saving these for bigger expenses.
